\documentclass[11pt]{amsart}

\usepackage[adobe-utopia]{mathdesign}	
\let\circledS\undefined

\usepackage{geometry}               
\usepackage{amsrefs} 				
\usepackage[T1]{fontenc}
\geometry{letterpaper}
\usepackage{amssymb, color}
\usepackage[toc,page]{appendix}

\usepackage{fancyhdr}
\setlength{\headheight}{13pt}
\usepackage{lastpage} 

\usepackage{xypic}
\usepackage{enumerate} 

\usepackage{fancyvrb}	
\usepackage[usenames,dvipsnames]{xcolor}
\fvset{frame=single,framesep=1mm,fontfamily=tt,fontsize=\scriptsize,numbers=left,framerule=.3mm,numbersep=1mm,commandchars=\\\{\}}

\usepackage{graphicx}
\usepackage{longtable}		% Long table
\usepackage[colorlinks = true, citecolor = red]{hyperref}

\newcommand*{\myTagFormat}[2]{(\cref{#1})($#2$)}
\newcommand{\eqname}[1]{\tag*{#1}} 
\renewcommand\qedsymbol{$\clubsuit$}

\fancyhf{} 
\rfoot{Page \thepage\ of \pageref{LastPage}}
\fancyhead[LE]{\leftmark} 
\fancyhead[RO]{\rightmark} 

\pagestyle{fancy}
\bibliographystyle{amsalpha}

%%%%%%%%%%%%%%%%%%%%%%%%%%%%%%%%%%%%%%%%%%%%%%%%%%%%%%%%
%     Math macros
%%%%%%%%%%%%%%%%%%%%%%%%%%%%%%%%%%%%%%%%%%%%%%%%%%%%%%%%

\def\A{\mathbb A}
\def\C{\mathbb C}
\def\Q{\mathbb Q}
\def\R{\mathbb R}
\def\Z{\mathbb Z}
\def\AAA{\mathcal A}	% Automorphic forms
\def\CCC{\mathcal C}
\def\DDD{\mathcal D}
\def\FFF{\mathcal F}
\def\LLL{\mathcal L}
\def\MMM{\mathfrak M}	% (G,M) family of intertwining operators 
\def\PPP{\mathcal P}
\def\O{\mathcal O}
\def\o{\scalebox{0.8}{$\scriptstyle\mathcal{O}$}}
\def\aaa{\mathfrak a}
\def\cb#1{{\color{blue}#1}}
\def\d{\text d}
\def\K{\textbf K}
\def\Ad{\operatorname{Ad}}
\def\Aut{\operatorname{Aut}}
\def\bs{\setminus} 			% had \backslash earlier
\def\det{\operatorname{det}}
\def\dim{\operatorname{dim}}
\def\Gal{\operatorname{Gal}}
\def\GL{\operatorname{GL}}
\def\geom{\text{geom}}
\def\Hom{\operatorname{Hom}}
\def\Id{\operatorname{Id}}
\def\Ind{\operatorname{Ind}}
\def\l{\ell}
\def\Lone{L^1}
\def\Ltwo{L^2}
\def\lmod#1{\left\lvert #1 \right\vert} % Large mod
\def\Lieg{\mathfrak g}
\def\mod#1{\vert #1 \vert} % absolute value, mod
\def\norm#1{\Vert #1 \Vert} % norm
\def\RRR{\mathcal R}
\def\Re{\operatorname{Re}}
\def\Res{\operatorname{Res}}
\def\se{\subseteq}
\def\sprod#1#2{\left\langle #1 , #2 \right\rangle}  % pairing
\def\trace{\operatorname{trace}}
\def\Vol{\operatorname{Vol}}

\newtheorem{theorem}{Theorem}[section]
\newtheorem{corollary}[theorem]{Corollary}
\newtheorem{lemma}[theorem]{Lemma}

\theoremstyle{remark}
\newtheorem{remark}[theorem]{Remark}

%%%%%%%%%%%%%%%%%%%%%%%%%%%%%%%%%%%%%%%%%%%%%%%%%%%%%%%%

\begin{document}

\title{List of corrections / typos in Arthur's Clay notes}
\author{ Abhishek Parab}
\email{abhishekparab@gmail.com}
\date{\today}                       % Activate to display a given date or no date

\maketitle

This is a list of corrections and typographical errors in Arthur's Clay notes. If you notice typos not included here, please email me at \href{mailto:abhishekparab@gmail.com}{abhishekparab@gmail.com}. Any errors in these errata is entirely my fault. Notation: ``p.x l.y" refers to line y from the top on page x (not counting headers) whereas ``p.x l.-y" refers to line y counted from the bottom (counting lines in footnotes).

\hfill Abhishek\\[2em]

\begin{longtable} {ll}
	p. 12 l. 18: & Replace $G(F_S)$ with $G(\cb{\Q_S})$. \\[0.5em]
	p. 35 l. 4: & The definition of intertwining operator should have the term $\text{e}^{-(s\lambda + \rho_{P'})(H_{P'}(x))}$ \\ &  inside the integral. \\[0.5em]	
	p. 35 l. -15:  & The line should read ${\color{blue}x \mapsto} \displaystyle \sum_{P \in \PPP} n_P^{-1} \int_{i\aaa_P^*} E(x, F_P(\lambda), \lambda) \d \lambda$. \\[0.5em]
	p. 38 l. -1: & $\tau_P(H_P(\delta x))$ should be replaced with $\tau_P(H_P(\delta x) \cb{- T})$. \\[0.5em]
	p. 42 Figure 8.5: & The point labelled $H$ is $\cb{H_1}$. \\[0.5em]
	p. 44 l. 10: & The matrix $\begin{pmatrix} u_1 & 0 \\ 0 & u_1^{-1} \end{pmatrix}^{-1}$ should be replaced with $\begin{pmatrix} u_1 & \cb{*} \\ 0 & u_1^{-1} \end{pmatrix}^{-1}$. \\[0.5em]
	p. 46 l. 13: & Replace $Q$ with $\cb{P}$. \\[0.5em]
	p. 46 l. 16: & Replace $P$ with $\cb{Q}$. \\[0.5em]
	p. 55 l. 3: & The integral should be replaced with a \cb{sum}. \\[0.5em]
	p. 59 l. -4: & Replace $G(\Q)_\gamma$ with $G(\Q)_{\cb{\gamma_1}}$ at both places. \\[0.5em]
	p. 61 l. 3: & Replace $f(x^{-1} \gamma x)$ with $f(x^{-1} \cb{\gamma_1} x)$. \\[0.5em]
	p. 61 l. 6: & $\chi_T$ should be $\cb{\psi_T}$. \\[0.5em]
	p. 66 l. -6: & The $\PPP$ should be replaced with \cb{$P$} in the index of the direct sum.\\[0.5em]
	p. 67 l. 5, 6: & The $\PPP$ in $\mathcal B_{\PPP, \chi}$ should be replaced with \cb{$P$} at both places. \\[0.5em]
	p. 67 l. -13: & The integral is over $i\aaa_{\cb{P_1}}^*$. \\[0.5em]
	p. 73 l. -7: & The integral is missing $\cb{\d n}$. \\[0.5em]
	p. 75 l. 14: & Replace $=$ with $\cb{\leq}$. \\[0.5em]
	p. 82 l. -6: & The term $H_Q(\delta x)$ should be replaced with $H_Q(\cb{y})$ in Equation (15.5). \\[0.5em]
	p. 97 l. -3: & The sum is taken over $\cb{Q \in \FFF(M)}$. \\[0.5em]
	p. 101 l. -4: & Replace $c_M^{Q_2}(\lambda)$ with $\cb{d}_M^{Q_2}(\lambda)$. \\[0.5em]
	p. 101 l. -2: & Replace $d_M^{Q_1}$ with $d_M^{\cb{Q_2}}$. \\[0.5em]
	p. 105 l. 1: & The line should read ``is the point \cb{in} $K_S$ such that $\dots$". \\[0.5em]
	p. 105 l. 13: & The integral is taken over $M_{Q, \gamma}(F_S)\bs M_Q(F_S)$. \\[0.5em]
	p. 109 l. -20: & The last sum needs to be taken over $Q \in \FFF(L)$. \\[0.5em]
	p. 115 l. 8: & The reference is [\textbf{\cb{A12}}, Corollary 8.7] and not \textbf{A11}. \\[0.5em]
	p. 139 l. -6: & $I_M(\gamma, f)$ to be replaced by $\cb{J}_M(\gamma, f)$. \\[0.5em]
	p. 142 l. -16: & The integrand should read $h(y) (R_{y^{-1}}f)_{\cb{Q, y}} \d y$. \\[0.5em]
	p. 147 l. 7: & $\cdots$ space of functions on $\cb{\prod_{\text{temp}}}(G(F_S))\times \aaa_{G, S}$. \\[0.5em]
	p. 147 l. 14: & The function $\phi$ is on the space $\cb{\prod_{\text{temp}}}(G(F_S))\times \aaa_{G, S}$. \\[0.5em]
	p. 149 l. 14: & The equation referred to should be (\cb{22.13}) and not (23.13). \\[0.5em]
	p. 151 l. -7: & The sum should be taken over $M \in \cb{\LLL}$. \\[0.5em]
	p. 232 l. -7: & The integral is missing a $\cb{\d \pi}$. \\[1em]
%	p. l. : & \\[0.5em]
%	p. l. : & \\[0.5em]
%	p. l. : & \\[0.5em]
%	p. l. : & \\[0.5em]
%	p. l. : & \\[0.5em]
	
\end{longtable}

\end{document}
