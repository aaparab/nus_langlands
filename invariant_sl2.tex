\documentclass[11pt]{amsart}

\usepackage[adobe-utopia]{mathdesign}	
\let\circledS\undefined

\usepackage{geometry}               
\usepackage{amsrefs} 				
\usepackage[T1]{fontenc}
\geometry{letterpaper}
\usepackage{amssymb, color}
\usepackage[toc,page]{appendix}

\usepackage{fancyhdr}
\setlength{\headheight}{13pt}
\usepackage{lastpage} 

\usepackage{xypic}
\usepackage{enumerate}
\usepackage{tikz}

\usepackage{fancyvrb}	
\fvset{frame=single,framesep=1mm,fontfamily=tt,fontsize=\scriptsize,numbers=left,framerule=.3mm,numbersep=1mm,commandchars=\\\{\}}

\usepackage{empheq}
\newcommand*\widefbox[1]{\fbox{\hspace{2em}#1\hspace{2em}}}

\usepackage{graphicx}
\usepackage[hidelinks]{hyperref}
\usepackage[capitalise, noabbrev]{cleveref}

\renewenvironment{rcases}		% \begin{cases} \end{cases} but to right. 
  {\left.\begin{aligned}}
  {\end{aligned}\right\rbrace}

\fancyhf{} 
\rfoot{Page \thepage\ of \pageref{LastPage}}
\fancyhead[LE]{\leftmark} 
\fancyhead[RO]{\rightmark} 

\pagestyle{fancy}
\bibliographystyle{amsalpha}

%%%%%%%%%%%%%%%%%%%%%%%%%%%%%%%%%%%%%%%%%%%%%%%%%%%%%%%%
%     Math macros
%%%%%%%%%%%%%%%%%%%%%%%%%%%%%%%%%%%%%%%%%%%%%%%%%%%%%%%%

\def\apg{a_{P} - a_{G}}
\def\A{\mathbf A}
\def\C{\mathbf C}
\def\I{\mathbf I}
\def\Q{\mathbf Q}
\def\R{\mathbf R}
\def\Z{\mathbf Z}
\def\AAA{\mathcal A}	% Automorphic forms
\def\BBB{\mathcal B}
\def\CCC{\mathcal C}
\def\DDD{\mathcal D}
\def\FFF{\mathcal F}
\def\HHH{\mathcal H}
\def\III{\mathcal I}
\def\LLL{\mathcal L}
\def\MMM{\mathfrak M}	% (G,M) family of intertwining operators 
\def\PPP{\mathcal P}
\def\SSS{\mathcal S}
\def\XXX{\mathcal X}
\def\O{\mathcal O}
\def\o{\scalebox{0.8}{$\scriptstyle\mathcal{O}$}}
\def\myshift{\mathcal S}	% My shift operator
\def\UUU{\mathcal U}
\def\aaa{\mathfrak a}

\def\d{\text d}
\def\K{\textbf K}
\def\Ad{\operatorname{Ad}}
\def\Aut{\operatorname{Aut}}
\def\bs{\setminus}
\def\cent{\operatorname{Cent}}
\def\cusp{\text{cusp}}
\def\det{\operatorname{det}}
\def\disc{\text{disc}}
\def\dim{\operatorname{dim}}
\def\Gal{\operatorname{Gal}}
\def\gl{\operatorname{GL}}
\def\geom{\text{geom}}
\def\hom{\operatorname{Hom}}
\def\Id{\operatorname{Id}}
\def\Im{\operatorname{Im}}
\def\Ind{\operatorname{Ind}}
\def\l{\ell}
\def\Lone{L^1}
\def\Ltwo{L^2}
\def\lmod#1{\left\lvert #1 \right\vert} % Large mod
\def\Lieg{\mathfrak g}
\def\mod#1{\lvert #1 \rvert} % absolute value, mod
\def\norm#1{\Vert #1 \Vert} % norm
\def\nnn{\mathfrak n}
\def\oP{\overline{P}}
\def\RRR{\mathcal R}
\def\Re{\operatorname{Re}}
\def\reg{\operatorname{reg}}
\def\Res{\operatorname{Res}}
\def\se{\subseteq}
\def\sl{\operatorname{SL}}
\def\so{\operatorname{SO}}
\def\sprod#1#2{\left\langle #1 , #2 \right\rangle}  % pairing
\def\tild{\widetilde}
\def\trace{\operatorname{trace}}
\def\unit{\operatorname{unit}}
\def\vol{\operatorname{Vol}}

\newtheorem{theorem}{Theorem}[section]

\newtheorem{corollary}[theorem]{Corollary}
\newtheorem{lemma}[theorem]{Lemma}
\newtheorem{example}[theorem]{Example}

\theoremstyle{remark}
\newtheorem{remark}[theorem]{Remark}

%%%%%%%%%%%%%%%%%%%%%%%%%%%%%%%%%%%%%%%%%%%%%%%%%%%%%%%%

\begin{document}

\title{The Invariant Trace Formula for SL(2)}
\author{Abhishek Parab}
\email{abhishekparab@gmail.com}
\date{\today}                       % Activate to display a given date or no date

\maketitle
\setcounter{tocdepth}{1}		% Shows chapters in toc (table of contents) but not sections

\begin{abstract}
	These notes are aimed to get a working understanding of the terms involved in the Arthur-Selberg trace formula for the group $\sl(2)$ in the adelic setting closely following the excellent notes \cite{clay} by Prof. Arthur in the Clay volume. We analyze the truncated kernel and develop the coarse and fine geometric and spectral expansions for $\sl(2)$. The invariant trace formula is also computed explicitly for $\sl(2)$. We end with a brief note on recent convergence results. 
\end{abstract}

\tableofcontents

%%%%%%%%%%%%%%%%%%%%%%%%%%%%%%%%%%%%%%%%%%%%%%%%%%%%%%%%
%     Introduction
%%%%%%%%%%%%%%%%%%%%%%%%%%%%%%%%%%%%%%%%%%%%%%%%%%%%%%%%
\section{Introduction} \label{sec_intro}
	
	An important purpose of these notes is to motivate the reader to understand Arthur's Clay notes \cite{clay} which are an excellent introduction to the trace formula. We will abbreviate and say trace formula when we refer to one of the versions of the Arthur-Selberg trace formula, the non-invariant, invariant or stable depending on the context. 
	
	In its easiest form the trace formula is the expression of the trace of the right regular representation of a group $H$ on the space $\Ltwo(\Gamma\bs H)$ where $\Gamma$ is a co-compact subgroup of $H$, when expressed in two different ways. It is obtained by integrating the kernel over the diagonal subgroup. Selberg developed a trace formula for $\sl(2)$ and used it to prove the existence of Maass forms. Thereafter it has been vastly generalized by Arthur in the adelic setting for a connected reductive group. Understanding it for $\sl(2)$ is a primary objective of these notes. In the general case, the kernel is not integrable so Arthur modifies it to obtain the truncated kernel, an alternating sum indexed by standard parabolic subgroups of the group. This truncated kernel has two important properties, namely that its integral over the diagonal converges absolutely and that it agrees with the (usual) kernel of the right regular representation on a compact set. Arthur develops the coarse geometric and spectral expansions by refining the expression for this truncated kernel. For certain special classes on the two sides he gives more explicit forms. He then goes on to make the remaining terms in these expansions more explicit and refers to it as the fine expansion. 
	
	Soon after Jacquet and Langlands \cite{MR0401654} used the trace formula to compare representations of $\gl(2)$ and its twisted forms, it became clear that one of the crucial uses of the trace formula would be to prove functoriality by comparing trace formulas on different groups. This could well be the motivation for Arthur to develop and refine the trace formula as well as the seminar of Clozel-Labesse-Langlands in developing the twisted trace formula (for connected components of reductive algebraic groups). Having developed the fine expansions, Arthur develops the invariant version by transferring the non-invariant terms on the spectral side to the geometric side. For most groups (including $\sl(2)$), the transfer of orbital integrals involves a matching of not just invariant orbital integrals but stable ones. He refines the invariant trace formula to get the stable version. Assuming the fundamental lemma (now proven), he then goes ahead to prove functoriality for classical groups which can be considered as one of the monumental achievements of this theory. 

	Although many endoscopic cases of functoriality are proven and we deduce more information about the parameters involved, important non-endoscopic cases like symmetric powers still remains open. It was Langlands' paper in `Beyond Endoscopy' \cite{MR2058622} that galvanized work in this direction. Very naively the hope now is to be able to define (completed) automorphic $L$-functions by imitating the method of Godement-Jacquet (theory of monoids developed by Braverman-Kazhdan, Ngo and Vinberg) and develop a trace formula with spectral side weighted by the residues of these $L$-functions. To prove the `Beyond Endoscopy' cases of functoriality one hopes to be able to compare the geometric sides of two such trace formulas. 
	
	We begin these notes by reviewing the co-compact case and discussing Arthur's modified kernel and its properties. Although it is very instructive to go over the proofs in \cite{clay}*{S 8, 9} we will restrict to discussing a few geometric and combinatorial ideas that go into the proof. Always equipped with the example of $\sl(2)$ we will discuss the coarse geometric and spectral expansions. We then sketch the fine expansions and the invariant trace formula and end the notes with a brief mention of recent convergence results with conjectural applications to Beyond Endoscopy. 
	
	Throughout these notes we assume the base field to be $\Q$ for notational simplicity but the complete theory has been developed for any number field. 

%%%%%%%%%%%%%%%%%%%%%%%%%%%%%%%%%%%%%%%%%%%%%%%%%%%%%%%%
%     The co-compact case
%%%%%%%%%%%%%%%%%%%%%%%%%%%%%%%%%%%%%%%%%%%%%%%%%%%%%%%%
\section{The co-compact case} \label{sec_cpt}

In this section we will develop the trace formula when the quotient is compact. One reason for including the details of the co-compact case is to see that the simplest terms occurring in the non-co-compact case are precisely the ones occurring here. 

Let $H$ be a locally compact unimodular topological group and $\Gamma$ be a discrete (not necessarily co-compact in $H$) subgroup of $H$. An important question is to decompose the right regular representation 
\begin{align*}
	R : & H \to \gl(\Ltwo(\Gamma \bs H)), \\
	(R(y) & \phi)(x) = \phi(xy), \qquad \phi \in \Ltwo(\Gamma\bs H)
\end{align*}
into irreducible unitary representations. 
\begin{example}
	Take $H = \R$ and $\Gamma = \Z$. The irreducible unitary representations of $H$ are $(x \mapsto \exp(\lambda x)) $ for $\lambda \in i\R.$ The isomorphism 
	\[ R \cong \displaystyle\bigoplus_{\lambda \in 2\pi i \Z} \exp(\lambda x) \]
	can be realized via the Plancherel theorem for Fourier series namely,
	\begin{align*}
		\Ltwo(\Z \bs \R) & \overset{\sim}{\longrightarrow} \Ltwo(\Z) \\
		\phi & \mapsto \hat\phi, \qquad \text{where }\hat\phi(n) = \int_{\Z\bs \R} \phi(x) \exp(2\pi i n x) \d x
	\end{align*}
	Plancherel's theorem : $\norm{\phi} = \norm{\hat\phi}$. 
\end{example}

\begin{example}
	When we take $\Gamma = \{1\}$ in the previous example the decomposition of $R$ is continuous and is given by the Plancherel theorem for Fourier transforms, namely $\norm{f}_2 = \norm{\hat f}_2$ where $\hat f$ is the Fourier transform of $f$. 
\end{example}

In general for arbitrary $H$ and $\Gamma, R$ decomposes into a discrete and a continuous spectrum. Langlands' theory of Eisenstein series gives an explicit decomposition in terms of the cuspidal spectrum and we will briefly review when dealing with the spectral expansion. In order to study the representation $R$, we would like to understand the trace of the operator $R(f)$ on $\Ltwo(\Gamma\bs H)$ for a compactly supported function $f$ on $H$ where
\[ R(f) = \int_H f(y) R(y)\ \d y. \]
\begin{align*}
	(R(f)\phi)(x) & = \int_H f(y) (R(y)\phi)(x) \d x \\
			& = \int_H f(y) \phi(xy) \d y \\
			& = \int_H f(x^{-1} y) \phi(y) \d y \\
			& = \int_{\Gamma\bs H} \sum_{\gamma \in \Gamma} f(x^{-1}\gamma y) \phi(y) \d y, \qquad \text{since } \phi(\gamma y) = \phi(y)\\
			& = \int_{\Gamma\bs H} \phi(y) K(x, y) \d y
\end{align*}
where $K(x, y) = \sum_{\gamma \in \Gamma} f(x^{-1}\gamma y)$. The sum over $\gamma$ is finite since $f$ is of compact support. 

To continue further, we make the very special assumption: $\Gamma\bs H$ is compact. As a consequence, the following hold.
\begin{enumerate}
\item 	\begin{itemize}
		\item[] $K(x, y)$ is compactly supported 
		\item[] Hence square-integrable
		\item[] $\Rightarrow R(f)$ is Hilbert-Schmidt class
		\item[] $\Rightarrow R(f)$ is compact (self-adjoint) operator. 
		\item[] Therefore by the spectral theory of self-adjoint compact operators,
			\[ R = \bigoplus_\pi m(\pi, R) \pi \]
			where $0 \leq m(\pi, R) < \infty$. Additionally if we assume 
			\[ f(x) = (g*g^*)(x) = \int_H g(y) \overline{g(x^{-1}y)} \d y\]
			for a function $g$ on $H$ of compact support then $R$ is self-adjoint. 
		\end{itemize}
\item If $H$ is a Lie group and $f$ is smooth of compact support then $R(f)$ is of trace class so
	\[ \trace R(f) = \int_{\Gamma\bs H} K(x, x) \d x. \]
\end{enumerate}

Suppose $\{\Gamma\}$ is a set of representatives of conjugacy classes in $\Gamma$. For any subset $\Omega$ of $H$, let $\Omega_\gamma$ denote the centralizer of $\gamma$ in $\Omega$. Then, 
\begin{flalign*}
	\trace R(f) &= \int_{\Gamma\bs H} K(x, x) \ \d x \\
				&= \int_{\Gamma\bs H} \sum_{\gamma \in \Gamma} f(x^{-1}\gamma x) \ \d x \\
				&= \int_{\Gamma\bs H} \sum_{\gamma \in \{\Gamma\} } 
						\sum_{\delta \in \Gamma_\gamma\bs \Gamma} f(x^{-1}\delta^{-1}\gamma\delta x) \ \d x \\
				&= \int_{\Gamma_\gamma \bs H} \sum_{\gamma \in \{\Gamma\}} f(x^{-1} \gamma x) \ \d x \\
				&= \sum_{\gamma \in \{\Gamma\}} \int_{H_\gamma\bs H} \int_{\Gamma_\gamma \bs H_\gamma} 
						f(x^{-1}u^{-1}\gamma u x) \d u \ \d x \\
				&= \sum_{\gamma \in \{\Gamma\}} \vol(\Gamma_\gamma \bs H_\gamma) \int_{H_\gamma \bs H} 
						f(x^{-1}\gamma x) \d x \qquad \text{since } u \in H_\gamma \text{ so } 
						u^{-1}\gamma u = \gamma. 
\end{flalign*}
This is the geometric expansion. On the other hand, the decomposition
\[ R = \bigoplus m(\pi, R) \pi \]
gives
\[ \trace R(f) = \sum_\pi m(\pi, R) \trace \pi(f). \]
Thus we have an identity of linear forms, 
\[ \displaystyle \sum_\gamma a_\Gamma^H(\gamma) f_H(\gamma) = \sum_\pi a_\Gamma^H(\pi) f_H(\pi), \]
where $\gamma \in \{\Gamma \}$, 
\[	\begin{rcases}
		f_H(\gamma) = \int_{H_\gamma \bs H} f(x^{-1} \gamma x) \d x \quad \\
		a_\Gamma^H(\gamma) = \vol(\Gamma_\gamma \bs H_\gamma) \quad
	\end{rcases} \quad
	\text{Geometric side}\\[0.5em] \]
	
\[	\text{Spectral side} \quad 
	\begin{cases}
		\quad f_H(\pi) = \trace \pi(f) = \trace \left( \int f(y) \pi(y) \d y \right) \\
		\quad a_\Gamma^H(\pi) = m(\pi, R).
	\end{cases}
\]
This is the Selberg trace formula for compact quotient. As a quick exercise the reader should use this formula to prove Frobenius reciprocity when $H$ is a finite group. Also, when $H = \R$ and $\Gamma = \Z$ it is easy to see the trace formula reduces to the familiar Poisson summation formula, 
\[ \sum_{n \in \Z} f(n) = \sum_{n \in \Z} \hat f(n). \]


%%%%%%%%%%%%%%%%%%%%%%%%%%%%%%%%%%%%%%%%%%%%%%%%%%%%%%%%
%     Notations
%%%%%%%%%%%%%%%%%%%%%%%%%%%%%%%%%%%%%%%%%%%%%%%%%%%%%%%%
\section{Notations} \label{sec_not1}

Before investigating the problems we run into when generalizing the compact case, we introduce some notations. 

Let $G$ be a connected reductive group over $\Q$ and denote by $\A$ the adeles of $\Q$. For concreteness it is good to have an explicit group in mind, like $\gl(3)$ or $\text{Sp}(4)$. Throughout the notes, we will explicitly carry out calculations for $G = \sl(2)$. Let $A_G$ be the largest central subgroup of $G$ over $\Q$ that is a $\Q$-split torus. (So $A_G \cong \gl(1)^k$). In the case of $\sl(2), A_G = \{1\}$. Denote by $X(G)_\Q = X(G)$ the free abelian group of rank $k$ given by
\[ X(G) = \hom_\Q(G, \gl(1)). \]
Define the real vector spaces
\begin{align*}
	& \aaa_G := \hom_\Z(X(G), \R) \\
	& \aaa_G^* := X(G) \otimes_\Q \R.
\end{align*}
and their respective complexification by $\aaa_{G, \C}$ and $\aaa_{G, \C}^*$. 
Define the Harish-Chandra map by
\begin{align*}
	H_G : G(\A) & \to \aaa_G \\
	\sprod{H_G(x)}{\chi} & = \log \mod{\chi(x)}, \qquad \chi \in X(G)
\end{align*}
and denote its kernel by $G(\A)^1$. If we denote $A_G(\R)^\circ$ by $\AAA_G$ then $G(\A)$ is the direct product of $G(\A)^1$ and $\AAA_G$. When $G = \sl(2)$, the vector spaces $\aaa_G, \aaa_G^*$ and the map $H_G$ are all trivial and $G(\A) = G(\A)^1$.

If the reader is not familiar with the notion of a parabolic subgroup they may consider the example of $\sl(2)$ wherein the group of upper triangular matrices is the unique (up to conjugation) parabolic subgroup. In general, fix a minimal parabolic subgroup $P_0$ with Levi decomposition $P_0 = M_0 N_0$. Call a parabolic subgroup $P$ as \underline{standard} if $P \supseteq P_0$.  Such a parabolic subgroup has a unique Levi decomposition ($M_P \supseteq M_0$) given by the exact sequence
\[ 1 \to N_P \to P \to M_P \to 1. \]
The Weyl group is the finite quotient $N_G(M_0)/Z_G(M_0)$ denoted by $W = W^G$. In the case of $\sl(2)$, it is customary to choose the set of upper triangular matrices with determinant 1 as the minimal parabolic subgroup $P_0$. The only standard parabolic subgroups are $\{P_0, G\}$.
Let $A_P, \AAA_P, \aaa_P, \aaa_P^*$ denote $A_{M_P}, A_{M_P}(\R)^\circ, \aaa_{M_P}$ and $\aaa_{M_P}^*$ respectively. If $P = P_0$ we will further abbreviate to $A_0, \AAA_0, \aaa_0, \aaa_0^*$ etc. Given standard parabolic subgroups $P, Q$ the \underline{Weyl set} denoted by $W(\aaa_P, \aaa_Q)$ is the set of $\R$-linear isomorphisms from $\aaa_P$ to $\aaa_Q$ obtained b restricting the elements of $W^G$ to $\aaa_P \subseteq \aaa_0$. 

The trace formula we develop essentially depends on the choice of a maximal compact subgroup of $G(\A)$ which we now choose. For $G = \sl(2)$ and $p$ a rational prime, denote $K_p = \sl(2, \Z_p)$. At the Archimedean place, let $K_\infty = \so(2, \R)$. Then $K = K_\infty \times \prod_p K_p$ is a maximal compact subgroup of $\sl(2, \A)$. In general for every rational prime $p$, we fix $K_p$ to be a maximal compact subgroup of $G(\Q_p)$ satisfying certain conditions (i.e., corresponds to a special point in the Bruhat-Tits building; Arthur calls them as `good'). Having defined the maximal compact subgroup $K$ of $G(\A)$, we extend the map $H_P = H_{M_P}$ initially defined on $M_P(\A)$ to $G(\A)$ by setting
\begin{align*}
	H_P & : G(\A) \to \aaa_P \\
	H_P(g) & = H_P(m) = H_{M_P}(m)
\end{align*} 
where we use the Iwasawa decomposition to write $g = nmk$ with $n \in N_P(\A), m \in M_P(\A)$ and $k \in K$. 

Arthur discusses two problems when we mimic the case of $(H, \Gamma)$ of the previous section when the quotient $\Gamma\bs H$ is not compact. The geometric side would look like 
\[ \sum_{\gamma \in \{G(\Q)\}} \vol(G(\Q)_\gamma\bs G(\A)_\gamma^1) \int_{G(\A)^1_\gamma \bs G(\A)^1} f(x^{-1}\gamma x) \d x.\]

\begin{itemize}
\item[Problem 1:] $\vol(G(\Q)_\gamma\bs G(\A)_\gamma^1)$ may be infinite. 
\item[Problem 2:] The integral over $G(\A)^1_\gamma \bs G(\A)^1$ may diverge. 
\end{itemize}

Arthur explains these divergence issues for $G = GL(2)$ and attributes them to the existence of nontrivial parabolic subgroups. Indeed we have

\begin{theorem} [Borel-Harish--Chandra]
The quotient $G(\Q)\bs G(\A)^1$ is non-compact if and only if $G$ has proper parabolic subgroups defined over $\Q$. 
\end{theorem}

One of Arthur's first contribution is to truncate the kernel by writing it as an alternating sum over standard parabolic subgroups of $G$ and prove it converges absolutely. We discuss this in the next section. 

%%%%%%%%%%%%%%%%%%%%%%%%%%%%%%%%%%%%%%%%%%%%%%%%%%%%%%%%
%     Truncation of the kernel
%%%%%%%%%%%%%%%%%%%%%%%%%%%%%%%%%%%%%%%%%%%%%%%%%%%%%%%%
\section{The kernel and its truncation}

To explain the terms in the truncated kernel we need to define some notations. Suppose we have two standard parabolic subgroups $P_1 \se P_2$. Thus there is a $\Q$-rational embedding
\[ A_{P_2} \se A_{P_1} \se M_{P_1} \se M_{P_2}. \]
The restriction homomorphism 
\[ X(M_{P_2})_{\Q} \to X(M_{P_1})_{\Q} \]
gives a linear injection
\[ \aaa_{P_2}^* \hookrightarrow \aaa_{P_1}^* \]
and a dual linear surjection
\[ \aaa_{P_1} \twoheadrightarrow \aaa_{P_2}. \]
We denote the kernel of the latter map by $\aaa_{P_1}^{P_2}$. The homomorphism $X(A_{P_1})_{\Q} \to X(A_{P_2})_{\Q}$ is surjective so gives a surjection
\[ \aaa_{P_1}^* \twoheadrightarrow \aaa_{P_2}^* \]
and a dual linear injection
\[ \aaa_{P_2} \hookrightarrow \aaa_{P_1}. \]
Thus we have a split exact sequence of real vector spaces, namely
\[ 0 \to \aaa_{P_1}^{P_2} \to \aaa_{P_1} \leftrightarrows \aaa_{P_2} \to 0\]
and 
\[ 0 \to \aaa_{P_2}^* \rightleftarrows \aaa_{P_1}^* \to \aaa_{P_1}^* / \aaa_{P_2}^* \to 0. \]
Set $(\aaa_{P_1}^{P_2})^* := \aaa_{P_1}^* / \aaa_{P_2}^*$. 

For any parabolic subgroup $P$, let $\Phi_P$ denote the set of roots of $(P, A_P)$. Identify $\Phi_P$ as a subset of $\aaa_P^*$ by 
\[ \Phi_P \se X(A_P)_\Q \se X(A_P)_\Q \otimes \R = \aaa_P^*. \]
Set $\Phi_0 := \Phi_{P_0}$. This is a valid root system. Let $\Delta_0 \se \Phi_0$ denote the set of simple roots. Then $\Delta_0$ is a basis of $(\aaa_0^G)^*$ as a real vector space. Analogously the set $\Delta_0^\vee = \{ \alpha^\vee : \alpha \in \Delta_0\}$ of coroots is a basis of $\aaa_0^G:= \aaa_{P_0}^G$. Denote the dual bases of $\Delta_0$ (resp. $\Delta_0^\vee$) by $\hat\Delta_0^\vee$ (resp. $\hat\Delta_0$). 

By the theory of algebraic groups there is a bijection between subsets $\Delta_0^P$ of $\Delta_0$ and standard parabolic subgroups $P$ of $G$ over $\Q$ such that 
\[ \aaa_P = \{ H \in \aaa_0 : \alpha(H) = 0 \ \forall \alpha \in \Delta_0^P \}. \]
Denote by $\Delta_P$ the set of linear forms on $\aaa_P$ obtained by restricting elements of $\Delta_0 \bs \Delta_0^P$. It is a basis of $(\aaa_P^G)^* := \aaa_P^* / \aaa_G^*$. Another basis is $\hat\Delta_P = \{ \varpi_\alpha : \alpha \in \Delta_0 \bs \Delta_0^P \}.$ The corresponding dual bases are 
\[ \hat\Delta_P^\vee = \{\varpi_\alpha^\vee : \alpha \in \Delta_P \} \]
and
\[ \Delta_P^\vee = \{ \alpha^\vee : \alpha \in \Delta_P \}. \]

More generally when $P_1 \se P_2$ we define the subsets
\[ \Delta_{P_1}^{P_2}, \hat\Delta_{P_1}^{P_2} \subset \aaa_{P_1}^{P_2} \]
and
\[ (\Delta_{P_1}^{P_2})^\vee, (\hat\Delta_{P_1}^{P_2})^\vee \subset (\aaa_{P_1}^{P_2})^* \]
analogously, with `everything happening inside $M_{P_2}$'. Note that the notion of roots and co-roots of a root system is true when $P_1 = P_0$ but not in general. 


We now calculate these objects for $\sl (2)$. As remarked earlier, the set $\PPP^G(M_0)$ of standard parabolic subgroups of $G$ with Levi contained in $M_0$ is $\{G, P_0\}$. Since $\sl(2)$ does not have nontrivial characters so $X(\sl(2))$ is trivial. The real vector spaces $\aaa_G$ and $\aaa_G^*$ are also trivial. However $X(M_0)$ is spanned by the root $\beta_1 = e_1 - e_2$ where $e_i\begin{pmatrix} t_1 & \\ & t_2 \end{pmatrix} = t_i$ and $t_1 t_2 = 1$. 
Thus, we have
\begin{align*}
	 \text{Roots: } & \Delta_0 = \{\beta_1 = e_1 - e_2 \}, \\
	 \text{Co-roots: } & \Delta_0^\vee = \{ \beta_1^\vee = e_1^\vee - e_2^\vee \}, \\
	 \text{Weights: } & \hat\Delta_0 = \{ \varpi_1 = \frac{1}{2}(e_1 - e_2) \}, \text{ and } \\
	 \text{Co-weights: } & \hat\Delta_0^\vee = \{ \varpi_1^\vee = \frac{1}{2}(e_1^\vee - e_2^\vee) \},
\end{align*}
where the usual relations $(e_i, e_j) = \delta_{i, j}$ and $(\beta_i, \beta_i^\vee) = 2$ hold. 

Recall that $K$ was chosen to be $\so(2, \R) \times \displaystyle\prod_p \sl(2, \Z_p)$. Then we can identify elements in the adelic quotient 
\[ \sl(2, \Q) \bs \sl(2, \A) / K \]
as points in the fundamental domain for $\sl(2, \Z)$ via
\begin{equation} \label{fund_dom} 
\sl(2, \Q) \bs \sl(2, \A) / K \ \simeq \ \sl(2, \Z) \bs \sl(2, \R) / \so(2, \R) \ 
		\simeq \ \sl(2, \Z) \bs \mathbf H,
\end{equation}
where $\mathbf H$ is the upper half complex plane $\{ x + i y : x, y \in \R, \ y > 0\}$. We have the Iwasawa decomposition $g = nmk$ as 
\[ x + i y = \begin{pmatrix} 1 & x \\ 0 & 1 \end{pmatrix} \begin{pmatrix} y^{1/2} & \\ & y^{-1/2} \end{pmatrix}
		\begin{pmatrix} \cos \theta & \sin \theta \\ -\sin \theta & \cos \theta \end{pmatrix} . i \]
The map $H_0 : G(\A) \to \aaa_0$ satisfies

\[ \sprod{H_0(g)}{\beta_1} = \log \ \mod{\beta_1(g)\ } = \log \ \mod{\beta_1 
			\begin{pmatrix} y^{1/2} & \\ & y^{-1/2} \end{pmatrix} \ } = \log \mod{y}. \]

For a standard parabolic subgroup $P$, let $\tau_P$ be the characteristic function of the cone 
\[ \aaa_P^+ = \{ T \in \aaa_P : \alpha(T) > 0 \ \forall \alpha \in \Delta_P. \} \]
Analogously let $\hat\tau_P$ be the characteristic function of the subset
\[ \{ T \in \aaa_P : \varpi(T) > 0 \ \forall \varpi \in \hat\Delta_P \}. \]
We say the point $T$ is `sufficiently' regular if for every $\alpha \in \Delta_0, \alpha(T) \gg 0$. This means that the point $T$ is in the positive Weyl chamber sufficiently away from the walls in $\aaa_0$. 

When $G = \sl(2)$ when $P = P_0$, these cones are just rays on the line $\aaa_0$. The point $T \in \aaa_0$ is regular if it is sufficiently away from the origin in the positive direction. Although case of $\sl(2)$ simplifies the combinatorics, the example of $\sl(3)$ that Arthur carries out is quite instructive. 

Just as we had the right regular representation $R = R_G$ of $G(\A)$ on $\Ltwo(G(\Q) \bs G(\A))$ so also for every parabolic subgroup $P = M_P N_P$, the regular representation $R_P$ of $G(\A)$ on $\Ltwo(N_P(\A) M_P(\Q) \bs G(\A))$ is defined by 
\[ (R_P(y)\phi)(x) = \phi(xy). \]
Indeed, 
\[ R_P = \Ind_{N_P(\A)M_P(\A)}^{G(\A)} \mathbf 1_{N_P} \otimes R_{M_P}. \]
This gives an operator $R_P(f)$ for $f \in \CCC_c^\infty(G(\A))$ whose kernel is given by
\[ K_P(x, y) = \int_{N_P(\A)} \sum_{\gamma \in M_P(\Q)} f(x^{-1} \gamma n y) \d n, \qquad 
		x, y \in N_P(\A)M_P(\Q)\bs G(\A). \]

Arthur defines the modified kernel for $T \in \aaa_0$ sufficiently regular (depending on $f$) as
\begin{equation} \label{kernel}
	k^T(x) = k^T(x, f) = \sum_{P \supseteq P_0} (-1)^{\apg} \sum_{\delta \in P(\Q) \bs G(\Q)}
		K_P(\delta x, \delta x) \hat\tau_P(H_P(\delta x) - T). 
\end{equation}

For $G = \sl(2)$ there are two terms given by
\begin{equation} \label{k_sl2}
	k^T(x, f) = K_G(x, x) - \sum_{\delta \in P_0(\Q) \bs G(\Q)} K_0(\delta x, \delta x) \ \hat\tau_0(H_0(\delta x) - T).
\end{equation}

The main steps that Arthur undertakes in the development of the trace formula are:
\begin{itemize}
	\item proves the integral over the kernel, namely
			\[ J^T(f) = \int_{G(\Q) \bs G(\A)^1} k^T(x, f) \d x \]
			converges absolutely for $T$ sufficiently regular,
	\item shows that the map $T \mapsto J^T(f)$ is a polynomial in $T \in \aaa_0$, and
	\item gets the spectral and geometric expansions out of $k^T(x)$. 
\end{itemize}
Because it covers many important aspects of the trace formula, we will go into details discussing the proof of Theorem 6.1 for $\sl(2)$. For instance, the coarse geometric expansion follows closely on the steps of the proof of this theorem. The combinatorics discussed here play an important role in the geometric expansion. 

%%%%%%%%%%%%%%%%%%%%%%%%%%%%%%%%%%%%%%%%%%%%%%%%%%%%%%%%
%     Proof of Theorem 6.1
%%%%%%%%%%%%%%%%%%%%%%%%%%%%%%%%%%%%%%%%%%%%%%%%%%%%%%%%
\section{Proof of Theorem 6.1 for $\sl(2)$} \label{6.1proof}

In the case of $G = \sl(2)$, consider the characteristic function $\hat\tau_0(H_0(\delta x) - T)$ appearing in \cref{k_sl2}. For what values of $\delta x$ does it equal 1?

Fix $x \in \sl(2, \A)$ and recall the identification of $\sl(2, \Q)\bs \sl(2, \A) / K$ in \cref{fund_dom} with the fundamental domain of $\sl(2, \Z)$. Clearly the image of $x$ and $\delta x$ in this fundamental domain agree when $\delta \in P_0(\Q) \bs G(\Q)$. 
Also recall that $\hat\Delta_0 = \{\varpi_1 = \frac{1}{2}(e_1 - e_2) \}$ and $\hat \tau_0$ is the characteristic function of the subset 
\[ \{ T \in \aaa_0 : \varpi_1(T) > 0 \}. \]
We can write $T \in \aaa_0$ as $T = t \beta_1^\vee$ with $t \in \R$ and $\Delta_0 = \{\beta_1\}$. The condition that $T$ is sufficiently regular just means that $t \gg 0$. Now the condition that $\hat\tau_0 (H_0(\delta x) - T) = 1$ is equivalent to 
\[ \varpi_1(H_0(\delta x)) = \varpi_1(H_0(x)) > \varpi_1(T), \]
which implies
\[ \log \mod{\varpi_1(x)} > \varpi_1(t \beta_1^\vee) = t. \]
Identifying $x \in \sl(2, \Q)\bs \sl(2, \A) / K$ with the point $x + iy$ in the fundamental domain (note the unfortunate abuse of notation) we have
\begin{equation} \label{x_decom}
	\begin{aligned} 
		H_0(x) & = H_0(\begin{pmatrix} 1 & x \\ 0 & 1 \end{pmatrix} \begin{pmatrix} y^{1/2} & 0 \\ 0 & y^{-1/2} \end{pmatrix} \begin{pmatrix} \cos \theta & \sin \theta \\ -\sin \theta & \cos \theta \end{pmatrix}) \\
			& = H_0(\begin{pmatrix} y^{1/2} 0 & \\ 0 & y^{-1/2} \end{pmatrix}). \\
	\therefore \log & \mod{\varpi_1(\begin{pmatrix} y^{1/2} & 0 \\ 0 & y^{-1/2} \end{pmatrix})} > t \\
	\therefore y & > \exp(2t).
	\end{aligned}
\end{equation}

\begin{figure} 
\centering
	\begin{tikzpicture}
		\draw [gray] (1,2)  arc[radius = 2, start angle= 60, end angle= 120];
		\draw [gray] (1, 2) -- (1,5);	% right vertical line
		\draw [gray] (-1, 2) -- (-1,5); % left vetical line
		\draw [->] (-3, 0.3) -- (3, 0.3) node[right] {\tiny $Y$-\text{axis}}; % Y-axis
%		\draw (2.5, 4) node[right] {$y = \exp(2t)$};
		\draw (0, 3) node {$P = G$};
		\draw (0, 4.5) node {$P = P_0$};
		\draw (-1.5, 1.5) -- (-1.5, 5);
		\draw (1.5, 1.5) -- (1.5, 5);
		\draw (-1.5, 1.5) -- (1.5, 1.5) node[right] {$T_1 : y = \exp(t_1)$};
		\draw (0, 1.3) node {$\omega$};
		\draw [dashed] (-1.5, 4) -- (1.5, 4) node[right] {$T : y = \exp(2t)$};
	\end{tikzpicture}
\caption{Partitions of the Siegel set} \label{fig:sd}
\end{figure}

As noted earlier in \cref{k_sl2}, the truncated kernel for $\sl(2)$ is
\begin{equation} \label{k_sl2}
	k^T(x, f) = K_G(x, x) - \sum_{\delta \in P_0(\Q) \bs G(\Q)} K_0(\delta x, \delta x) \ \hat\tau_0(H_0(\delta x) - T).
\end{equation}
Observe that if $x$ belongs to the lower half in \cref{fig:sd} then
\[ k^T(x) = K_G(x, x) = K(x, x) = \sum_{\gamma \in \sl(2, \Q)} f(x^{-1}\gamma x). \]
This is true in general, there is a compact set such that $k^T(x)$ equals $K_G(x, x)$ for $x$ in this compact set. 

\subsection{Siegel sets}
Suppose $T_1 \in \aaa_0$ and $\omega$ is a compact subset of $N_{P_0}(\A)M_{P_0}(\A)^1$. The subset 
\begin{flalign*}
	\SSS^G(T_1) = \SSS^G(T_1, \omega) = \{ x = pak \in G(\A) : p \in \omega, a \in A_0, k \in K \text{ such that } \tau_0(H_0(a) - T_1) = 1 \}
\end{flalign*}
is called the \underline{Siegel set} associated to $T_1$ and $\omega$. 

We would like to know what the condition $\tau_0(H_0(a) - T_1) = 1$ means for $G = \sl(2)$. The decomposition $M_0(\A) = M_0(\A)^1 \times \AAA_0$ is given by the norm map on the ideles and is equivalent to $\I_\Q = \I^1 \times (\R^*)^0$. As before, we can write $a = \begin{pmatrix} y^{1/2} & \\ & y^{-1/2} \end{pmatrix}$ and $\Delta_0 = \{\beta_1 = e_1 - e_2\}$. Write $T_1 = t_1 \varpi_1^\vee$. Then
\[	\tau_0(H_0(a) - T_1) = 1 \Leftrightarrow \beta_1(H_0(a)) > \beta_1(T_1) \Leftrightarrow \log \mod{y} > t_1, \]
that is, $y > \exp(t_1)$.
\begin{theorem} [Borel--Harish-Chandra]
One can choose $T_1$ and $\omega$ so that 
\[ G(\A) = G(\Q) \SSS^G(T_1, \omega). \]
\end{theorem}

For this to hold in the case of $\sl(2)$, we ought to cover the fundamental domain. So the compact subset $\omega \subseteq N_0(\A) M_0(\A)^1$ must be chosen of width greater than that of the fundamental domain, i.e., width $> 1$ and $T_1 = t_1 \omega_1$ should satisfy $\exp(t_1) < 1/2$. Now onward fix $T_1$ and $\omega$ satisfying this theorem. Define the \underline{truncated Siegel set} for $T \in \aaa_0$ by
\[ \SSS^G(T, T_1, \omega) = \{ x \in \SSS^G(T_1, \omega) : \varpi(H_0(x) - T) \leq 0 \ \forall \varpi \in \hat\Delta_0. \} \]
Write $F^G(x, T)$ to be the characteristic function in $x$ of the projection of $\SSS^G(T_1, T, \omega)$ onto $G(\Q)\bs G(\A)$. More generally for a standard parabolic subgroup $P$ define 
\[ \SSS^P(T_1) = \SSS^P(T_1, \omega), \; \SSS^P(T_1, T) = \SSS^P(T_1, T, \omega) \text{ and } F^P(x, T) \]
by replacing $\Delta_0, \hat\Delta_0$ and $G(\Q)\bs G(\A)$ with $\Delta_0^P, \hat \Delta_0^P$ and $P(\Q)\bs G(\A)$ in the respective definitions. We have the partition lemma of Arthur:
\begin{lemma} \label{partition_lemma}
	For any $x \in G(\A)$, 
	\[ \sum_{P \supseteq P_0} \sum_{\delta \in P_0(\Q) \bs G(\Q)} F^P(\delta x, T) \tau_P(H_P(\delta x) - T) = 1 \]
\end{lemma}
For a geometric interpretation of this lemma, see \cite{clay}*{p. 39}. In the case of $\sl(2)$ the content of this lemma is that the fundamental domain for $\sl(2)$ is partitioned according to standard parabolic subgroups as in \cref{fig:sd}.

We now begin the proof of \cite{clay}*{Theorem 6.1} for $G=\sl(2)$. There are many simplifications since $\sl(2)$ has rank one. In particular,
\[ k^T(x) = K_G(x, x) - \sum_{\delta \in P_0(\Q)\bs G(\Q)} \hat\tau_0(H_0(\delta x) - T) K_0(\delta x, \delta x). \]
We multiply the first term above by the LHS of \cref{partition_lemma} which yields (for $P=G$),
\[ F(x, T) + \sum_{\delta \in P_0(\Q)\bs G(\Q)} \tau_0(H_0(\delta x) - T) = 1. \]
\begin{flalign*} 
	\therefore k^T(x) = F(x, T) K_G(x, x) \ +  
	\sum_{\delta \in P_0(\Q)\bs G(\Q)} \left( 
		\tau_0(H_0(\delta x) - T) K_G(\delta x, \delta x)
		- \hat\tau_0(H_0(\delta x) - T) . K_{P_0}(\delta x, \delta x) \right).
\end{flalign*}
The first term (corresponding to $P=G$),
\[ F(x, T) K_G(x, x) = F(x, T) \sum_{\gamma \in G(\Q)} f(x^{-1}\gamma x) \]
is nonzero when $F(x, T)=1$, i.e., when the image of $x$ is in a compact subset of the Siegel set. Thus the integral over $G(\Q) \bs G(\A)^1$ of this term converges absolutely. It only remains to analyze the second term. Since $\tau_0(H-T) = 1$ if and only if $\hat\tau_0(H-T) = 1$ we may as well assume this to be the case. Thus we only need to consider the integral over $x \in G(\Q)\bs G(\A)$ for which $\tau_0(H_0(x) - T) = \hat\tau_0(H_0(x) - T) = 1$. Thus $x$ lies in the ``upper'' unbounded subset of the Siegel domain of \cref{fig:sd}. We would like to show the integral
\begin{flalign*}
	\int_{G(\Q)\bs G(\A)^1} \sum_{\delta \in P_0(\Q)\bs G(\Q)} \tau_0(H_0(\delta x) - T) \left(K_G(x, x) - K_0(\delta x, \delta x) \right) \d x \\
	= \int_{P_0(\Q)\bs G(\A)^1} \tau_0(H_0(x) - T) \Big( K_G(x, x) - K_0(x, x) \Big) \d x
\end{flalign*}
converges absolutely. Write $x = p_0 a_0 k$ where $p_0 \in P_0(\Q) \bs M_0(\A)^1 N_0(\A), a_0 \in A_0(\R)^\circ$ and $k \in K$. Since $p_0$ and $k$ belong to compact sets, it suffices to consider the integral over $a_0 \in A_0(\R)^\circ$ such that 
\[ \tau_0(H_0(a_0)-T) = 1. \]

Let 
\begin{flalign*}
	K_{P_0, G}(x) & := K_G(x, x) - K_0(x, x) \\
			& = \sum_{\mu \in G(\Q)} f(x^{-1}\mu x) - \sum_{\gamma \in M_0(\Q)} \int_{N_0(\A)} 
				f(x^{-1} \gamma n x) \d n.
\end{flalign*}
Arthur uses the compactness of $f$ to show the first sum over $G(\Q)$ can in fact be taken over $P_0(\Q)$. He carries out the example computation for $G = \sl(2)$ on p. 44 in \cite{clay}. Thus, 
\[ k_{P_0, G}(x) = \sum_{\mu \in M_0(\Q)} \left[\sum_{\nu \in N_0(\Q)} f(x^{-1} \mu \nu x) -
		\int_{N_0(\A)} f(x^{-1} \mu n x) \d n \right]. \]
Using the isomorphism $\exp : \nnn_0 \xrightarrow{\sim} N_0$, one replaces the sum (resp. integral) over the group $N_0$ by a sum (resp. integral) over $\nnn_0$ which would allow us to apply Poisson summation formula. Thus,
\[ k_{P_0, G}(x) = \sum_{\mu \in M_0(\Q)} \left[ \sum_{\zeta \in \nnn_0(\Q)} f(x^{-1} \mu \exp(\zeta) x) -
		\int_{X \in \nnn_0(\A)} f(x^{-1} \mu \exp(X) x) \d X.		
 \right] \]


Fixing a nontrivial additive character $\psi$ on $\A/\Q$ and a bilinear form on $\nnn_0(\A) \simeq \A$ and applying Poisson summation formula to the first term yields
\[ k_{P_0, G}(x) = \sum_{\mu \in M_0(\Q)} \sum_{\xi \in \nnn_0(\Q) \bs \{0 \}} \int_{\nnn_0(\A)}
		f(x^{-1} \mu \exp(X_1) x) \psi(\sprod{\xi, X_1}) \d X_1.
\]

It is here that the cancellation occurs due to the alternating sum over standard parabolic subgroups. For $\sl(2)$, this cancellation is reflected in the sum over $\xi \in \nnn_0(\Q) \bs \{0\}$. Since $f$ is of compact support so the function
\[ h_{x, \mu}(Y_0) = \int_{X_1 \in \nnn_0(\A)} f(x^{-1} \mu \exp(X_1) . \psi(\sprod{Y_1, X_1})) \d X_1 \]
is rapidly decreasing. It is a basic property of Fourier transforms that the $a_0$-component of $x = p_0 a_0 k$ acts on $\xi \in \nnn_0(\Q) \bs \{0\}$ by dilations. The function $h_{x, \mu}$ is finitely supported in $\mu \in M_0(\Q)$ again because of the compact support of $f$. Thus the integral of $k_{P_0, G}(x)$ over $P_0(\Q)\bs G(\A)$ converges absolutely. \qed

\begin{remark}
	Arthur says the integrand is compactly supported in $H_2 \in \aaa_{P_2}$ which does not seem to be the case. However following the steps in the proof of this theorem in \cite{duke}*{p. 947}, we only need the bound $\norm{H_2} \leq c \norm{H_1^2}$. 
\end{remark}

%%%%%%%%%%%%%%%%%%%%%%%%%%%%%%%%%%%%%%%%%%%%%%%%%%%%%%%%
%     The coarse geometric expansion
%%%%%%%%%%%%%%%%%%%%%%%%%%%%%%%%%%%%%%%%%%%%%%%%%%%%%%%%
\section{The geometric expansion} \label{sec:geom}

Before delving into the geometric expansion, we will review two important properties of the distribution $J^T(f)$. 
\begin{enumerate}
	\item For any $f \in \CCC_c^\infty(G(\A))$, the function
		\[ T \mapsto J^T(f) \]
		defined for $T \in \aaa_0$ sufficiently regular, is a polynomial in $T$ with degree $\leq a_0^G := \dim \aaa_0^G$. 
		
		Using this result we define $J(f)$ as $J^{T_0}(f)$ where $T_0 \in \aaa_0$ is a unique point such that the distribution $J^{T_0}(f)$ is independent of the choice of $P_0 \in \PPP(M_0)$, the set of minimal parabolic subgroups with Levi $M_0$. For $\sl(2)$ and in general for $\sl(n)$ and $\gl(n), T_0 \in \aaa_0$ is the origin.
		
	\item A distribution $I$ on $G(\A)$ is called \underline{invariant} if $I(f^y) = I(f)$ for every $y \in G(\A)$ where $f^y(x) = f(yxy^{-1})$. Arthur defines a map
	\[ \CCC_c^\infty(G(\A)) \to \CCC_c^\infty(M(\A)) \]
	given by
	\[ f \mapsto f_{Q, y} \]
	for any parabolic subgroup $Q$ containing $M$. We will not define the function $f_{Q, y}$ here (see \cite{clay}*{Theorem 9.4}) but remark that this is a natural way to restrict a function on $G$ to $M = M_Q$ by integrating over $K$ and $N_Q(\A)$. Although $J^T(f)$ is not invariant, it satisfies
	\[ J^G(f^y) = \sum_{Q \supseteq P_0} J^{M_Q}(f_{Q, y}). \]
\end{enumerate}

We now define the coarse conjugacy classes of Arthur. Recall that any element $\gamma \in G(\Q)$ has a Jordan decomposition
\[ \gamma = \gamma_s \gamma_u, \]
where $\gamma_s$ (resp. $\gamma_u$) is semisimple (resp. unipotent). Define two elements $\gamma, \gamma' \in G(\Q)$ to be \underline{$\O$-equivalent} if $\gamma_s$ and $\gamma_s'$ are conjugate over $G(\Q)$. Let $\O = \O^G$ denote the set of such equivalence classes which we will denote as \underline{coarse-conjugacy} or \underline{Arthur-conjugacy classes}. 

Clearly there is a bijection between coarse conjugacy classes and semisimple conjugacy classes in $G(\Q)$, namely 
\[ \o \in \O \mapsto [\gamma_s : \gamma \in \o ]. \]
Observe that if $1 \in \o$ then $\o$ consists of all unipotent elements in $G(\Q)$ and is known as the unipotent orbit (or unipotent variety) and denoted as $\UUU$. A class $\o \in \O$ is called \underline{anisotropic} if it does not intersect $P(\Q)$ for any proper parabolic subgroup in $G$ (not necessarily standard). 

\begin{lemma}
	An element $\gamma \in G(\Q)$ represents an anisotropic class if and only if the maximal $\Q$-split torus in the connected component of the centralizer $H$ of $\gamma$ in $G$ is $A_G$. 
\end{lemma}

Let us investigate the Arthur-conjugacy classes in $\sl(2)$. In the case of $\gl(n)$ they are in bijection with characteristic polynomials so for $\gl(2)$, every $\o \in \O^{\gl(2)}$ is one of the following types. Let $p$ be the characteristic polynomial of any semisimple element in $\o$. 
\begin{enumerate}
	\item $p$ is irreducible over $\Q$ and splits into distinct roots in a quadratic extension $L$ of $\Q$.
	\item $p$ factors into distinct roots over $\Q$, say $\gamma = \begin{pmatrix} a & \\ & b \end{pmatrix}: 
			a, b \in \gl(1)$,
	\item $p$ has a unique root in $\Q^*$, say $\gamma = \begin{pmatrix} a & \\ & a \end{pmatrix}: a \in \gl(1)$,
\end{enumerate}
When we look at $\sl(2)$ if two classes $\o, \o' \in \O^{\sl(2)}$ have different characteristic polynomials then $\o \neq \o'$. On the other hand we would like to analyze the case when $\o \neq \o'$ but they have the same characteristic polynomial. Fix a class $\o \in \O^{\sl(2)}$ and let $p$ be the associated characteristic polynomial. We have the three possibilities. 
\begin{enumerate}
	\item Suppose $p$ is irreducible over $\Q$ and factors in a quadratic extension $L$ of $\Q$. Suppose $\gamma_1$ and $\gamma_2$ are semisimple elements with characteristic polynomial $p$. They are conjugate over $\gl(2, \Q)$ say $\gamma_2 = g \gamma_1 g^{-1}$. To see whether they are conjugate over $\sl(2, \Q)$, consider the following equivalent statements. 
	\begin{itemize}
		\item There is an $h \in \sl(2, \Q)$ such that $\gamma_2 = h \gamma_1 h^{-1}$,
		\item There is a $c$ in the centralizer $\gl(2, \Q)_{\gamma_1}$ of $\gamma_1$ such that $\det c = \det g$. 
	\end{itemize}
		The relation $c = gh^{-1}$ proves this equivalence. Since $\det(\gl(2, \Q)) = \Q^*$, we would like to know the image of the map
    		\[ \det : \gl(2, \Q)_{\gamma_1} \to \Q^*, \]
		where $\gl(2)(\Q)_{\gamma_1}$ denotes the centralizer of $\gamma_1$ in $\gl(2, \Q)$. Since $\gamma_1$ represents an anisotropic class, this map coincides with the norm map
		\[ N_{L/\Q} : L^* \to \Q^*. \]
		This is not surjective and in fact the index of the image in $\Q^*$ is infinite (see \cite{MR1068677}). Thus every such $\o \in \O^{\gl(2)}$ is a disjoint union of infinitely many classes $\o \in \O^{\sl(2)}$. Take one such class $\o \in \O^{\sl(2)}$ and $\gamma \in \o$, say $\gamma = \begin{pmatrix} 5 & 3 \\ 3 & 2 \end{pmatrix}$. The (connected) centralizer of $\gamma$ is an anisotropic torus so consists only of semisimple elements. In our example, 
		\[ H_\gamma = \left\{ \begin{pmatrix} a & b \\ b & a - b \end{pmatrix} : a, b \in \Q^*, a^2 - ab - b^2 = 1 \right\}. \]
		There is no $\Q$-split torus inside $H_\gamma$ so this equals $A_G$ and Arthur defines such classes as \underline{anisotropic}. They are the easiest to deal with, as we will see. 
	\item Suppose $p$ has distinct roots $t^{\pm 1}$ then $\gamma = \begin{pmatrix} t & 0 \\ 0 & t^{-1} \end{pmatrix} \in \o$. The connected centralizer $H_\gamma$ is $M_0(\Q)$. 
	\item If $p$ has a unique root, it must be $1$ or $-1$ so $\begin{pmatrix} 1 & 0 \\ 0 & 1 \end{pmatrix} \in \o$ or $\begin{pmatrix} -1 & 0 \\ 0 & -1 \end{pmatrix} \in \o$ but not both. Since these are central elements, the connected centralizer in each of these cases is the full group $\sl(2, \Q)$. These elements are the most difficult to define the weighted orbital integrals. 
\end{enumerate}

An \underline{anisotropic rational datum} is an equivalence class of pairs $(P, \alpha)$ where $P \supseteq P_0$ and $\alpha$ is an anisotropic conjugacy class in $M_P(\Q)$, the Levi subgroup of $P$ containing $M_0$. The equivalence relation is just conjugacy, i.e., $(P, \alpha) \sim (P', \alpha')$ if $\alpha = w_s \alpha' w_s^{-1}$ for some $s \in W(\aaa_P, \aaa_{P'})$. 

There is a bijection between 
\begin{align*}
	\text{anisotropic rational data} \; & \leftrightarrow \; \text{semisimple conjugacy classes in } G(\Q) \\
	(P, \alpha) \quad & \mapsto \quad [\gamma_s] : \gamma \in \alpha.
\end{align*}
To see the surjection, take $P$ to be a parabolic subgroup containing $\alpha$ minimally. In the three cases above, the anisotropic rational data are respectively $[(G, \alpha)]$ where $\alpha$ is the anisotropic conjugacy class in $G(\Q)$, $\left[ \left(M_0, \begin{pmatrix} t & 0 \\ 0 & t^{-1} \end{pmatrix} \right) \right]$ and $\left[ \left(M_0, \begin{pmatrix} \pm 1 & 0 \\ 0 & \pm 1 \end{pmatrix} \right)\right]$.

For general $G$ we can split the sum in the definition of the kernel
\[ K(x, y) = \sum_{\gamma \in G(\Q)} f(x^{-1} \gamma y) \]
according to Arthur-conjugacy classes and write
\[ K(x, y) = \sum_{\o \in \O} K_{\o}(x, y),\]
where
\[ K_{\o}(x, y) = \sum_{\gamma \in \o} f(x^{-1}\gamma y). \]
More generally for a standard parabolic subgroup $P_0$, we can similarly decompose the kernel $K_P$ of the operator $R_P$ acting on $\Ltwo(N_P(\A) M_P(\Q) \bs G(\A))$. Recall that
\[ K_P(x, y) = \sum_{\gamma \in M_P(\Q)} \int_{N_P(\A)} f(x^{-1} \gamma n y) \d n. \]
We write $K_P(x, y) = \sum_{\o \in \O} K_{P, \o} (x, y)$ where
\[ K_{P, \o}(x, y) = \sum_{\gamma \in M_P(\Q) \cap \o} \int_{N_P(\A)} f(x^{-1} \gamma n y) \d n. \]

Thus, $k^T(x) = \sum_{\o \in \O} k^T_{\o}(x)$, where
\[ k^T_{\o}(x) = \sum_{P \supseteq P_0} (-1)^{\apg} \sum_{\delta\in P(\Q)\bs G(\Q)} K_{P, \o}(\delta x, \delta x) \hat\tau_P(H_P(\delta x) - T). \]

An important consequence of \cite{clay}*{Theorem 6.1} is that the sum 
\begin{equation}
	\sum_{\o \in \O} \int_{G(\Q) \bs G(\A)^1} k_{\o}^T(x) \d x 
\end{equation}
converges absolutely. The proof for $\sl(2)$ discussed in \cref{6.1proof} goes through assuming the equality:
\[ P_0(\Q) \cap \o = (M_0(\Q) \cap \o) N_0(\Q) \]
which is a straightforward calculation.

Having defined $J_{\o}^T(f)$ as
\[ J_{\o}^T(f) = \int_{G(\Q) \bs G(\A)^1} k_{\o}^T(x) \d x, \]
one sees that its behavior as a function of $T$ is the same as that of $J^T(f)$. As before we choose the unique point $T_0$ which frees us from the choice of the minimal parabolic subgroup $P_0$ (but not $M_0$) and define 
\[ J_{\o}(f) := J^{T_0}_{\o}(f). \]
The coarse geometric expansion is given by
\begin{equation} \label{coarse_geom}
	J(f) = \sum_{\o \in \O} J_{\o}(f). 
\end{equation}

\begin{remark}
	The invariance formula for $J(f)$ holds for $J_{\o}(f)$ namely,
	\[ J_{\o}(f^y) = \sum_{Q \supseteq P_0} J_{\o}^{M_Q}(f_{Q, y}). \]
	Consider the map $\O^{M_Q} \to \O^G$; a class $\o \in \O$ does not lie in the image of this map for every $Q \supseteq P_0$ if and only if $\o$ is anisotropic. If so, $J_{\o}(f^y) = J_{\o}(f)$ and so $J_{\o}(f)$ is an invariant distribution. 
\end{remark}

For `generic' classes $\o \in \O$, Arthur gives an explicit description of $J_{\o}(f)$ in terms of weighted orbital integrals which we now briefly describe. There are two assumptions on $\o$. 
\begin{itemize}
	\item The class $\o$ consists entirely of semisimple elements. 
	\item It is \underline{unramified}, i.e., the centralizer $G(\Q)_\gamma$ is contained in $M_P(\Q)$ where $(P, \alpha)$ is the anisotropic rational datum attached to $\o$. 
\end{itemize}
The first condition is equivalent to having no nontrivial unipotent elements in the centralizer of $\gamma$ for any $\gamma \in \o$. This implies that the connected component $H(\Q)$ of the centralizer $G(\Q)_\gamma$ is contained in $M_P(\Q)$. The second condition is clearly stronger. To see an example where the first condition holds but not the second, consider the example of 
\[ \gamma = \begin{pmatrix} 1 & \\ & -1 \end{pmatrix} \in G = \operatorname{PGL}(2). \]
In this case, $G(\Q)_\gamma$ is the product of the minimal torus with a group of order two so $H(\Q) \subseteq M_P(\Q)$ but $G(\Q)_\gamma \not\subseteq M_P(\Q)$. 

The result of these two assumptions is that if $(P, \alpha)$ and $(P', \alpha')$ are two representatives of the unramified class $\o$, then there is a ``unique'' element in $W(\aaa_P, \aaa_{P'})$ mapping $\alpha$ to $\alpha'$. Arthur analyzes this case and proves for such classes that
\begin{equation} \label{jo_anis}
	J_{\o}(f) = \vol(M_P(\Q)_\gamma \bs M_P(\A)^1) \int_{G(\A)_\gamma \bs G(\A)} f(x^{-1} \gamma x) v_P(x) \d x, 
\end{equation}
where $\gamma$ is any element in the $M_P(\Q)$-conjugacy class $\alpha$ and $v_P(x)$ is the volume of the projection onto $\aaa_P^G$ of the convex hull of certain points. Let us investigate this invariant orbital integral for $\sl(2)$. 

Suppose $G = \sl(2)$ and let $\o \in \O$ be the class corresponding to an irreducible characteristic polynomial and the rational datum $[(G, \alpha)]$. It is easy to see that the two conditions above are satisfied. For any $\gamma \in \alpha$, the centralizer $H_\gamma$ (which is connected) is an anisotropic torus over $\Q$ so the space
\[ M_P(\Q)_\gamma\bs M_P(\A)_\gamma = G(\Q)_\gamma \bs G(\A)_\gamma^1 \]
is compact. Its volume is the first term above. The term $v_M(x)$ equals 1 when $M=G$. Thus the contribution of these classes to the geometric side is
\[ J_{\o}(f) = \vol(G(\Q)_\gamma \bs G(\A)_\gamma) \int_{G(\A)_\gamma \bs G(\A)} f(x^{-1} \gamma x) \d x. \]
This expression agrees exactly with that in the co-compact case. 

Suppose $\o$ corresponds to the case when the characteristic polynomial has distinct roots, say $t$ and $t^{-1}$. Assume $\sigma = \begin{pmatrix} t & \\ & t^{-1} \end{pmatrix} \in \o$. If $\gamma \in \o$ then its Jordan decomposition would be $\gamma = (g\sigma g^{-1}) . u$ where $u$ is unipotent and $g \in G(\Q)$. Then $g^{-1} u g$ would be in $M_0(\Q)$, the centralizer of $\sigma$. Thus $u=1$ so $\o$ consists of semisimple elements only. Moreover $G(\Q)_\sigma = M_0(\Q)$ so both conditions are satisfied and 
\begin{flalign*}
	J_{\o}(f) &=  \vol(M_0(\Q) \bs M_0(\A)^1) \int_{G(\A)_\gamma \bs G(\A)} f(x^{-1} \gamma x) v_{M_0}(x) \d x \\
			&= \vol(\Q^*\bs \I^1) \int_{M_0(\A) \bs G(\A)}  f(x^{-1} \gamma x) v_{M_0}(x) \d x.
\end{flalign*}

The map $v_{M_0}(x)$ is the volume of the projection onto $\aaa_P^G$ of the convex hull of 
\[ \{ -w^{-1} H_{P'}(n_w x) : w \in W(\aaa_P, \aaa_{P'}), P' \supseteq P_0 \}. \]
The points $H_{P'}(n_w x)$ lie in $\aaa_{P'}^G$ and $w^{-1}$ maps them into $\aaa_P^G$. The element $n_w$ represents a representative of $w$ in $G(\Q)$. When $G=\sl(2)$ then $P' = P_0$ and the two points are $-H_{P_0}(x)$ and $H_{P_0}(n_w x)$ where $n_w = \begin{pmatrix} 0 & 1 \\ -1 & 0 \end{pmatrix}$ is a representative of the long element. Thus,
\begin{equation} \label{eq:v0}
	v_0(x) = - H_0(x) - H_0(n_w x).
\end{equation}
It is an interesting exercise to use the decomposition in \cref{x_decom} to show that $v_0(x) \geq 0$.

In general the function $v_M(x)$ is the smooth function corresponding to the $(G, M)$-family, a combinatorial notion that plays an important role in both the geometric and spectral expansions. We briefly discuss it below. 

Let $\PPP(M)$ denote the set of standard parabolic subgroups of $G$ containing $M$. (This is different from the class $\PPP$ of associated parabolic subgroups of Langlands which we discuss in the next section.) A family 
\[ \{c_P(\lambda) : \lambda \in i\aaa_M^*, P \in \PPP(M)\} \]
is called a $(G, M)$-family if the functions $c_P(\lambda)$ and $c_{P'}(\lambda)$ agree for $\lambda$ in the common wall between $P$ and $P'$ whenever they share such a wall. Given such a $(G, M)$-family, we can naturally assign to it a smooth function as
\[ c_M(\lambda) = \sum_{P \in \PPP(M)} c_P(\lambda) \theta_P(\lambda)^{-1} \]
where $\theta_P(\lambda) = \vol(\aaa_M^G / \Z \Delta_P^\vee)^{-1} . \displaystyle\prod_{\alpha \in \Delta_P} \lambda(\alpha^\vee)$. The function $v_M(x)$ is the value as $\lambda \to 0$ of the smooth function
\[ \sum_{P \in \PPP(M)} v_P(\lambda, x) \theta_P(\lambda)^{-1} \]
corresponding to the $(G, M)$-family 
\[ \{ v_P(\lambda, x) = \exp(-\sprod{\lambda}{H_P(x)}) \}. \]
Let us see the pole cancellation by evaluating it for $\sl(2)$ for which $\PPP(M_0) = \{ P_0, \oP_0 = w_0 P_0 w_0^{-1} \}$. The space $\aaa_0$ is spanned by $\beta_1^\vee$ and by $\varpi_1^\vee = \beta_1 / 2$. Write $\lambda \in \aaa_0^*$ as $\lambda = \lambda_1 \varpi_1$. 
\[ \theta_0(\lambda) = \vol(\aaa_0/\Z \beta_1^\vee) . \sprod{\lambda}{\beta_1^\vee} = \lambda_1. \]
Also, $\theta_{\oP_0}(\lambda) = -\lambda_1$ as $\Delta_{\oP_0}^\vee = \{ -\beta_1^\vee\}$. 
Observe that 
\begin{flalign*}
v_{M_0}(x) &= \lim_{\lambda \to 0} \frac{v_{P_0}(\lambda, x)}{\theta_0(\lambda)}
						+ \frac{v_{{\overline P}_0}(\lambda, x)}{\theta_{{\overline P}_0}(\lambda)} \\
			&= \lim_{\lambda \to 0} \frac{\exp(-\sprod{\lambda}{H_{P_0}(x)})}{\sprod{\lambda}{\beta_1^\vee}}
						+ \frac{\exp(-\sprod{\lambda}{H_{\overline P_0}(x)})}{\sprod{\lambda}{-\beta_1^\vee}}.
\end{flalign*}
Since $H_{\overline P_0}(x) = H_{P_0}(n_w x)$ where $n_w$ is the representative of the nontrivial element of $W^G$, so the limit as $\lambda \to 0$ agrees with the earlier definitionof $v_{M_0}(x) = v_0(x)$ in \cref{eq:v0} . 

Finally we consider the case when $\o \in \O$ has repeated eigenvalues. Thus $\o = \UUU$ or $-\UUU$ and this is the most difficult case to handle. In order to express all $\o \in \O$, Arthur carries out the following:
\begin{enumerate}
	\item He defines weighted orbital integrals $J_M(\gamma, f)$ attached to unramified classes $\o \leftrightarrow [(M, \alpha)]$; $\gamma \in \alpha$ \cite{MR625344}. 
	\item When $\gamma \in M(\Q_S)$ is arbitrary, he replaces $\gamma$ by $a \gamma$ for a point $a$ in general position, and shows that multiplication by a suitable function gives a finite limit as $a \to 1$ \cite{MR932848}.
	\item Having defined the weighted orbital integrals $J_M(\gamma, f)$ for general $\gamma$, he proves the geometric expansion for the unipotent variety of $G$, namely, $J_{\text{unip}}(f)$ in terms of the aforementioned weighted orbital integrals \cite{MR828844}. 
	\item Finally he inductively defines the fine expansion for general classes $\o \in \O$ by putting together the above objects \cite{MR835041}.
\end{enumerate}
Given $f \in \CCC_c^\infty(G(\A))$ there is a finite set $S$ of places containing the Archimedean place such that $f$ is a finite sum of functions whose component in $v \not \in S$ is the characteristic function of the maximal compact subgroup $K_v$ of $G(Q_v)$. Thus there is no loss in generality in assuming that $f \in \CCC_c^\infty(G(\Q_S))$. For unramified coarse conjugacy classes (and more generally whenever $G_\gamma = M_\gamma$), Arthur defines the weighted orbital integral as 
\begin{equation} \label{jm}
	J_M(\gamma, f) = \mod{D(\gamma)}^{\frac{1}{2}} \int_{G_\gamma(\Q_S)\bs G(\Q_S)} f(x^{-1}\gamma x) v_M(x) \d x
\end{equation}
where $D(\gamma)$ is the generalized Weyl discriminant. Using combinatorial identities associated with products of $(G, M)$-families, Arthur proves an expression for $J_M(\gamma, f)$ in terms of finite sums of products of local orbital integrals $J_M(\gamma_v, f_v)$. Thus $J_M(\gamma, f)$ is to be regarded as a local object. 

The expression for $J_M(\gamma, f)$ is not well-defined when $\o$ is not anisotropic. The extreme case of this is when $\o$ is the unipotent variety containing $\gamma = 1$ in $\sl(2)$. Here $v_P(x)$ is undefined and the integral in \cref{jm} does not converge. Arthur explains the solution he implements in this situation with an example of the group $\gl(2)$ but the same works for $\sl(2)$. Take $a = \begin{pmatrix} t & \\ & t^{-1} \end{pmatrix}$ with $t \neq \pm 1$ and calculate $J_{M_0}(a\gamma, f)$ where $\gamma = 1$. Arthur notes that adding a factor $r_{M_0}^G(a) = \log \mod{t^2 - t^{-2}}$ to $J_{M_0}(f)$ gives a locally integrable function around $a=1$. He defines
\begin{equation} \label{lim_orbint}
	J_{M_0}(1, f) = \lim_{a \to 1} J_{M_0}(a, f) + r_{M_0}^G(a) J_G(a, f).
\end{equation}

For general $G$, he shows there exist functions $r_M^L(\gamma, a)$ for Levi subgroups $L$ containing $M$, denoted as $L \in \LLL(M)$ such that the limit
\[ J_M(\gamma, f) := \lim_{a \to 1} \sum_{L \in \LLL(M)} r_M^L(a) J_L(a\gamma, f) \]
exists and equals the integral of $f$ with respect to a Borel measure on the set $\gamma^G$ (see \cite{clay}*{p. 103}). Having defined the weighted orbital integrals, Arthur proves the fine geometric expansion as
\begin{theorem} \cite{clay}*{Theorem 19.2}
	For any $\o \in \O$, there exists a set $S_{\o} \supseteq S_\infty$ such that if $S \supseteq S_{\o}$ and $f \in \CCC_c^\infty(G(F_S)^1)$ then
	\[ J_{\o}(f) = \sum_{M \in \LLL} \frac{\mod{W_0^M}}{\mod{W_0^G}} \sum_{\gamma \in (M(\Q)\cap \o)_{M, S}}
			a^M(S, \gamma) J_M(\gamma, f). \]
\end{theorem}

We use the above theorem to evaluate $J_{\o}(f)$ when $\o$ is the unipotent orbit in $\sl(2)$ denoted by $\UUU_G$. The case when $\o = (-1)*\UUU$ is similar. The term corresponding to $M=M_0$ is easily seen to be 
\[ \frac{1}{2} a_{M_0}(S, 1) J_{M_0}(1, f) = \frac{1}{2} \vol(Q^*\bs \I^1) J_{M_0}(1, f), \]
where $J_{M_0}(1, f)$ is as in \cref{lim_orbint}. The general definition of $(M, S)$-equivalence is somewhat involved but two elements $\gamma_1, \gamma_2 \in \UUU_G(\Q)$ are $(G, S)$-equivalent if they are conjugate over $G(\Q_S)$. Although there are infinitely many $G(\Q)$-conjugacy classes in $\UUU_G(\Q)$, the set $\UUU_G(\Q)_{G, S}$ of $(G, S)$-equivalence classes is finite. In fact the size of $(\UUU_G(\Q))_{G, S}$ is well-known from basic number theory to be either 1, 2, 4 or 8 depending on $S$. Indeed, two elements
\[ \gamma_1 = \begin{pmatrix} 1 & c_1 \\ 0 & 1 \end{pmatrix} \text{ and } \gamma_2 = \begin{pmatrix} 1 & c_2 \\ 0 & 1 \end{pmatrix}
\]
with $c_1, c_2 \in \Q^*$ are $G(\Q)$-conjugate if and only if $c_1 c_2^{-1} \in \Q^{*2}$. The contribution from $u=1$ is 
\[ \vol(G(\Q)\bs G(\A)^1) J_G(1, f). \]
Combining everything together we have, 
\[ J_{\text{unip}}(f) = \sum_{u \in (\UUU_G(\Q))_{G, S}} a^G(S, u) J_G(u, f) + \frac{1}{2} \vol(\Q^*\bs \I^1) J_{M_0}(1, f). \]

Putting all terms together, the expression for the fine geometric expansion for $\sl(2)$ is
\begin{equation} \label{eq:fine_geom}
	\begin{aligned}
		J_{\text{geom}}(f) = \sum_{\o \text{anisotropic}} & \vol(G(\Q)_\gamma \bs G(\A)^1_\gamma) 
					\int_{G(\A)_\gamma \bs G(\A)} f(x^{-1} \gamma x) \d x \\
			& + \sum_{\substack{\o : \text{char. poly.} \\ \text{distinct roots.}}} \vol(\Q^*\bs \I^1) 
					\int_{M_0(\A)\bs G(\A)} f(x^{-1}\gamma x) v_0(x) \d x \\
			& + \sum_{u \in (\UUU_G(\Q))_{G, S}} \left[ a^G(S, u) J_G(u, f) + a^G(S, -u) J_G(-u, f) 
					\right] \\
			& + \frac{1}{2} \vol(\Q^*\bs \I^1) \left[ J_{M_0}(1, f) + J_{M_0}(-1, f) 
					\right]. 
	\end{aligned}
\end{equation}

%%%%%%%%%%%%%%%%%%%%%%%%%%%%%%%%%%%%%%%%%%%%%%%%%%%%%%%%
%			The coarse spectral expansion     
%%%%%%%%%%%%%%%%%%%%%%%%%%%%%%%%%%%%%%%%%%%%%%%%%%%%%%%%
\section{The spectral expansion} \label{sec:spec}

In this section we discuss the spectral equivalent of \cref{coarse_geom,eq:fine_geom}. Unlike the co-compact case of the action of $H$ on $\Ltwo(\Gamma \bs H)$ of \cref{sec_cpt}, when $G$ is a connected reductive group, the representation $R_G$ of $G(\A)$ on $\Ltwo(G(\Q)\bs G(\A))$ does not decompose discretely. It is a direct sum of the discrete spectrum and the continuous spectrum which is described by Langlands' theory of Eisenstein series. An excellent reference is the book of M{\oe}glin and Waldspurger \cite{MW}, who refer to their book as \textit{Une Paraphrase de l'\'{E}criture}, a Paraphrase of the Scriptures. 

Henceforth we will assume that $f \in \CCC^\infty_c(G(\A))$ is left- and right-invariant under the `good` maximal compact subgroup $K$. The space of such functions will be called the \underline{Hecke algebra} and denoted by $\HHH(G)$.

Since $G(\A)$ is the direct product of $G(\A)^1$ and $\AAA_G = A_G(\R)^\circ$, given $\lambda \in \aaa_{G, \C}^*$ we can get a representation of $G(\A)$ on $\Ltwo(G(\Q) \bs G(\A))$ by
\[ R_{G, \disc, \lambda}(x) = R_{G, \disc}(x) \exp(\sprod{\lambda}{H_G(x)}), \qquad x \in G(\A). \]
Here, $R_{G, \disc}$ is the representation of $G(\A)^1$ to the subspace of $\Ltwo_\disc(G(\Q)\bs G(\A)^1)$ which decomposes discretely. It is unitary if and only if $\lambda \in i\aaa_G^*$. Suppose $P$ is a standard parabolic subgroup of $G$ and $\lambda \in \aaa_{P, \C}^*$. We write 
\[ y \mapsto \III_P(\lambda, y) \]
for the induced representation 
\[ \Ind_{P(\A)}^{G(\A)}(R_{M_P, \disc, \lambda} \otimes \mathbf 1_{N_P(\A)}). \]
The space for this representation is the space $\HHH_P$ of measurable functions
\[ \phi : N_P(\A)M_P(\Q)\AAA_P \bs G(\A) \to \C \]
such that 
\[ \norm{\phi}^2 = \int_K \int_{M_P(\Q)\bs M_P(\A)} \mod{\phi(mk)}^2 \d m \d k < \infty \]
and such that the function
\[ \phi_x : m \mapsto \phi(mx), \qquad \qquad m \in M_P(\Q)\bs M_P(\A)^1 \]
belongs to $\Ltwo_\disc(M_P(\Q) \bs M_P(\A)^1)$.
For any $y \in G(\A), \III_P(\lambda, y)$ maps the function $\phi \in \HHH_P$ to the function
\[ (\III_P(\lambda, y)\phi)(x) = \phi(xy) \exp(\sprod{\lambda + \rho_P}{H_P(xy)}) 
		\exp(\sprod{-(\lambda + \rho_P)}{H_P(x)}). \]
Indeed $\III_P(\lambda, y)$ is the representation induced from the right regular representation on $M$ twisted by $\lambda$ and the exponential factors are to ensure we land up in the right space after twisting. The operator $\III_P(\lambda, f)$ is defined in the usual manner. Let us investigate the space $\HHH_P$ for $P \in \PPP$ when $G = \sl(2)$. 
\begin{itemize}
	\item If $P=G$ then $\HHH_G = \Ltwo_{\disc}(G(\Q)\bs G(\A))$. 
	\item If $P = P_0$ then $M_0(\Q)\bs M_0(\A)^1 \simeq \Q^* \bs \I^1$. 
		Since this group is compact its spectrum is discrete. Moreover using the Iwasawa decomposition $G(\A) = P_0(\A) K$, we have
		\[ N_0(\A)M_0(\Q)\AAA_0\bs G(\A) \simeq M_0(\Q)\bs M_0(\A)^1 \times K \simeq \Q^*\bs \I^1 \times \operatorname{SO}(2). \]
		So if $\varphi \in \HHH_{P_0} = \HHH_0$ then $\varphi$ can be considered as a square-integrable function on the compact set $\Q^*\bs \I^1 \times \operatorname{SO}(2)$. 
\end{itemize}

Denote by $\HHH_P^\circ$ the subspace of $K$-finite vectors in $\HHH_P$. We say that two standard parabolic subgroups $P$ and $P'$ are \underline{associated} if the Weyl set $W(\aaa_P, \aaa_{P'})$ is non-empty and denote an equivalence class with respect to this relation as $\PPP$. In the case of $\sl(2)$ where there are two standard parabolic subgroups, namely $P_0, G$, it is easy to see that $W(\aaa_0, \aaa_G), W(\aaa_G, \aaa_G)$ and $W(\aaa_0, \aaa_0)$ have respectively 0, 1 and 2 elements. Thus there are two associated classes of parabolic subgroups, namely $\PPP = [P_0], [G]$. 

The spectral decomposition of Langlands gives an orthogonal direct sum decomposition
\[ \Ltwo(G(\Q)\bs G(\A)) \simeq \bigoplus_{\PPP} \Ltwo_\PPP(G(\Q)\bs G(\A)). \]

The term corresponding to $\PPP = [G]$ is the discrete spectrum. 
In the case of $\sl(2)$, the spectral decomposition is 
\[ \Ltwo(\sl(2, \Q)\bs \sl(2, \A)) = \Ltwo_{[G]} (\sl(2, \Q)\bs \sl(2, \A)) \oplus \Ltwo_{[P_0]} (\sl(2, \Q)\bs \sl(2, \A)) \]
and the term corresponding to $\PPP = [P_0]$ is the continuous spectrum.

For $x \in G(\A), \phi \in \HHH_P$ and $\lambda \in \aaa_{M, \C}^*$ the \underline{Eisenstein series} is defined as
\[ E(x, \phi, \lambda) = \sum_{\delta \in P(\Q) \bs G(\Q)} \phi(\delta x) 
		\exp(\sprod{\lambda + \rho_P}{H_P(\delta x)}). \]

Arthur describes the statement of the spectral decomposition in greater details in \cite{clay}*{Theorem 7.2} from which it follows that the kernel 
\[ K(x, y) = \sum_{\gamma \in G(\Q)} f(x^{-1}\gamma y) , \qquad f \in \HHH(G),\]
of the operator $R(f)$ has a formal expansion
\[ \sum_{P} n_P^{-1} \int_{i\aaa_P^*} \sum_{\phi \in \BBB_P} E(x, \III_P(\lambda, f)\phi, \lambda) 
		\overline{E(y, \phi, \lambda)} \d y \]
in terms of Eisenstein series. Here, 
\[ n_P = \sum_{P' \in \PPP = [P]} \mod{W(\aaa_P, \aaa_{P'})} \]
and $\BBB_P$ is a basis of $\HHH_P$ which is assumed to lie inside the dense subspace $\HHH_P^\circ$ of $K$-finite vectors. 

In addition, for every standard parabolic subgroup $Q$ we have an analogous expansion for the kernel 
\[ K_Q(x, y) = \int_{N_Q(\A)} \sum_{\gamma \in M_Q(\Q)} f(x^{-1}\gamma n y) \d n \]
of the operator $R_Q(f)$. Namely, we replace $n_P$ with $n_P^Q = n_{M_Q \cap P}$ and $E(x, \phi, \lambda)$ with \[ E_P^G(x, \phi, \lambda) = \sum_{\delta \in P(\Q) \bs Q(\Q)} \phi(\delta x) \exp(\sprod{\lambda + \rho_P}{H_P(\delta x)}). \]
We have, 
\[ K_Q(x, y) = \sum_{P \subseteq Q} (n_P^Q)^{-1} \int_{i\aaa_P^*} \sum_{\phi \in \BBB_P}
		E_P^Q(x, \III_P(\lambda, f)\phi, \lambda) \overline{E_P^Q(y, \phi, \lambda)} \d y. \]
The expression we obtain by substituting $K_Q(x, y)$ above in the truncated kernel, \cref{kernel}, is the starting point of the coarse spectral expansion. 

A function $\phi \in \Ltwo(G(\Q) \bs G(\A)^1)$ is called \underline{cuspidal} if 
\[ \int_{N_P(\A)} \phi(nx) \d n = 0 \]
for every proper parabolic subgroup in $G$ and almost every $x \in G(\A)^1$. The space of cuspidal functions is a closed $R_G$-invariant subspace of $\Ltwo(G(\Q)\bs G(\A)^1)$. Moreover we have
\begin{theorem} [Gelfand--Piatetski-Shapiro]
	\[ \Ltwo_\cusp(G(\Q)\bs G(\A)^1) \subseteq \Ltwo_\disc(G(\Q)\bs G(\A)^1), \]
	and the multiplicity of each irreducible representation is finite. 
\end{theorem}
Thus we get an orthogonal decomposition
\[ \Ltwo_\cusp(G(\Q)\bs G(\A)^1) = \bigoplus_\sigma \Ltwo_{\cusp, \sigma} (G(\Q)\bs G(\A)^1) \]
where $\sigma$ ranges over irreducible unitary representations of $G(\Q)\bs G(\A)^1$ and $\Ltwo_{\cusp, \sigma} (G(\Q)\bs G(\A)^1)$ is the $\sigma$-isotypic component, i.e., a direct sum of finitely many isomorphic copies of $\sigma$. 

We define a \underline{cuspidal automorphic datum} to be an equivalence class of pairs $(P, \sigma)$ where $P$ is a standard parabolic subgroup of $G$ and $\sigma$ is an irreducible unitary representation of $M_P(\A)^1$ such that the space $\Ltwo_{\cusp, \sigma} (M_P(\Q)\bs M_P(\A)^1)$ is nonzero. We say $(P, \sigma)$ and $(P', \sigma')$ are equivalent if there is an $s \in W(\aaa_P, \aaa_{P'})$ (with representative $w_s$) such that the representation 
\[ s^{-1}\sigma' : m \mapsto \sigma'(w_s m w_s^{-1}), \qquad m \in M_P(\A)^1 \]
is equivalent to $\sigma$. We write $\XXX = \XXX^G$ for the set of cuspidal automorphic data $\chi = [(P, \sigma)]$. 

When $P=G$ then $\sigma$ is a cuspidal automorphic representation of $\sl(2,\A)$ and when $P=P_0$ then $\sigma$ is an irreducible unitary representation of $M_0(\A)^1$ such that the space $\Ltwo_{\text{cusp}, \sigma}(M_0(\Q)\bs M_0(\A)^1) = \Ltwo(M_0(\Q)\bs M_0(\A)^1)$ is nonzero. In this case $\sigma$ corresponds to a Hecke character of $\Q^*\bs \I^1$. Two pairs $(P_0, \sigma_1)$ and $(P_0, \sigma_2)$ are equivalent if $\sigma_2 \simeq s.\sigma_1 = \sigma_1^{-1}$ where $s$ is the nontrivial element in $W = W^G$. Thus the cuspidal automorphic data for $\sl(2)$ are:
\begin{enumerate}
\item $[(G, \sigma)]$ where $\sigma$ is a cuspidal automorphic representation of $G$,
\item $[(P_0, \sigma), (P_0, \sigma^{-1})]$ where $\sigma$ is a character of $M_0(\A)^1$ trivial on $M_0(\Q)$ and $\sigma^2 \neq 1$, and
\item $[(P_0, \sigma)]$ where $\sigma$ is a character of $M_0(\A)^1$ trivial on $M_0(\Q)$ and $\sigma^2 = 1$.
\end{enumerate}

If $s \in W(\aaa_P, \aaa_{P'})$ we define the \underline{intertwining operator}
\[ M(s, \lambda) : \HHH_P \to \HHH_{P'} \]
by
\[ (M(s, \lambda)\phi)(x) = \int \phi(w_x^{-1}n x) \exp(\sprod{\lambda + \rho_P}{H_P(w_s^{-1}n x)})
				 \exp(\sprod{- s\lambda + \rho_{P'}}{H_{P'}(x)}) \d n. \]
As the name suggests, it intertwines $\III_P(\lambda)$ with $\III_{P'}(s\lambda)$ which is an equivalent way of saying the diagram below commutes. 
\[ \xymatrix{
	\HHH_P \ar[r]^{\III_P(\lambda)} \ar[d]_{M(s, \lambda)} & \HHH_P \ar[d]^{M(s, \lambda)} \\
	\HHH_{P'} \ar[r]^{\III_{P'}(s\lambda)} & \HHH_{P'}.
 } \]
Moreover,
\[ E(x, \III_P(\lambda, y)\phi, \lambda) = E(xy, \phi, \lambda). \]
These are the most important properties of Eisenstein series and intertwining operators. Formally they are easy to prove but to prove they converge and define meromorphic functions is very difficult.

Now we will define the decomposition
\[ \Ltwo(G(\Q)\bs G(\A)) \bigoplus_\PPP \Ltwo_{\PPP-\cusp}(G(\Q)\bs G(\A)) \]
which will lead us to the coarse spectral decomposition. Let $\HHH_{P, \cusp}$ be the subspace of vectors $\phi \in \HHH_P$ such that for almost every $x \in G(\A)$, the function
\[ \phi_x : m \mapsto \phi(mx) , \qquad m \in M(\A) \]
is in $\Ltwo_{\cusp}(M(\Q)\bs M(\A)^1)$. Then clearly,
\[ \HHH_{P, \cusp} = \bigoplus_\sigma \HHH_{P, \cusp, \sigma} \]
where $\HHH_{P, \cusp, \sigma}$ consists of functions in $\HHH_{P, \cusp}$ which transform according to $\sigma$. Suppose $\Psi(\lambda)$ is an entire function of $\lambda \in \aaa_{P, \C}^*$ of Paley-Wiener type, with values in a finite dimensional subspace of functions
\[ \{ x \mapsto \Psi(\lambda, x) \} \subseteq \HHH_{P, \cusp, \sigma}^\circ. \]
(Here, $\HHH_{P, \cusp, \sigma}^\circ$ is the intersection of $\HHH_{P, \cusp, \sigma}$ with $\HHH_P^\circ$.) Then, $\Psi(\lambda, x)$ is the Fourier transform in $\lambda$ of a smooth compactly supported function on $\aaa_P$, i.e., the function
\[ \psi(x) = \int_{\Lambda + i\aaa_P^*} \Psi(\lambda, x) \exp(\sprod{\lambda + \rho_P}{H_P(x)}) \d \lambda \]
of $x$ is compactly supported in $H_P(x) \in \aaa_P$. 
\begin{lemma} [Langlands]
	The function 
	\[ (E\psi)(x) := \sum_{\delta \in P(\Q)\bs G(\Q)} \psi(\delta x) \]
	is in $\Ltwo(G(\Q)\bs G(\A))$. Moreover if $\Psi'(\lambda', x)$ is another such function attached to a pair $(P', \sigma')$ then the inner product formula
	\[ (E\psi, E\psi') = \int_{\Lambda + i\aaa_P^*} \sum_{s \in W(\aaa_P, \aaa_{P'})}
				(M(s, \lambda)\Psi(\lambda), \Psi'(-s\overline \lambda)) \d \lambda \]
	holds for any point $\Lambda \in \aaa_P^*$ such that $\Lambda - \rho_P$ is in the positive Weyl chamber in $\aaa_P^*$. 
\end{lemma}

Langlands also proves an explicit formula for the inner product of Eisenstein series. This gives an orthogonal decomposition (see \cite{clay}*{p. 65})
\begin{equation} \label{hpchi}
	\Ltwo(G(\Q)\bs G(\A)) = \bigoplus_{\chi \in \XXX} \Ltwo_\chi(G(\Q)\bs G(\A))
\end{equation}
from which Arthur develops the coarse spectral expansion. Here $\Ltwo_\chi(G(\Q)\bs G(\A))$ is the closed $G(\A)$-invariant subspace of $\Ltwo(G(\Q)\bs G(\A))$ generated by the functions $E\psi$ attached to $(P, \sigma)$. To develop the fine expansion, he extends this inner product to truncated Eisenstein series. This is quite technical and is discussed in Sections 20, 21 of the Clay notes. 

For a standard parabolic subgroup $P$, the correspondence
\[ (P_1\cap M_P, \sigma_1) \mapsto (P_1, \sigma_1) \qquad P_1 \subseteq P, \; [(P_1 \cap M_P, \sigma_1)] \in \XXX^{M_P} \]
yields a mapping from $\XXX^{M_P}$ to $\XXX = \XXX^G$ which gives an orthogonal decomposition
\[ \HHH_P = \bigoplus_{\chi \in \XXX} \HHH_{P, \chi}. \]
Arthur claims the basis $\BBB_P$ of $\HHH_P$ assumed to lie in the dense subspace $\HHH_P^0$ respects the above decomposition in \cref{hpchi}, i.e., $\BBB_P = \coprod_{\chi \in \XXX} \BBB_{P, \chi}$. For any $\chi \in \XXX$ we set
\[ K_\chi(x, y) = \sum_P n_P^{-1} \int_{i\aaa_P^*} \sum_{\phi \in \BBB_{P, \chi}}
		E(x, \III_{P, \chi}(\lambda, f)\phi, \lambda) \overline{E(y, \phi, \lambda)} \d \lambda. \]
Then,
\[ K(x, y) = \sum_{\chi \in \XXX} K_\chi(x, y). \]
We repeat the procedure replacing $G$ with any standard parabolic subgroup and get the decomposition 
\[ K_P(x, y) = \sum_{\chi \in \XXX} K_{P, \chi}(x, y). \]
This gives the decomposition
\begin{align*}
	k^T(x) & = \sum_{\chi \in \XXX} k_\chi^T(x) \\
			& = \sum_P (-1)^{\apg} \sum_{\delta \in P(\Q)\bs G(\Q)} K_{P, \chi}(\delta x, \delta x)
				\hat\tau_P(H_P(\delta x) - T).
\end{align*}
To prove the convergence of 
\[ \int_{G(\Q)\bs G(\A)^1} \sum_{\chi \in \XXX} \mod{k_\chi^T (x)} \d x \]
is quite nontrivial and Arthur uses the truncation operator here. We will not go into the details but refer to Section 13 in the Clay notes. 

Similar to the geometric side, Arthur gives a more explicit formula for `generic' classes. He defines a class $\chi \in \XXX$ to be \underline{unramified} if for every pair $(P, \sigma) \in \chi$, the stabilizer of $\sigma$ in $W(\aaa_P, \aaa_P)$ is $\{1\}$. In the case of $\sl(2)$, the classes $[(G, \sigma)]$ are always unramified since $W(\aaa_G, \aaa_G) = \{1\}$. Moreover, when $\sigma$ is a character of $M_0(\Q)\bs M_0(A)^1$ such that $\sigma^2 \neq 1$ then $s.\sigma = \sigma^{-1}$ so the class $\chi$ attached to such $(P_0, \sigma)$ is unramified. 

For $J_\chi(f)$ where $\chi$ is an unramified classes corresponding to $(P, \sigma)$, Arthur gives an explicit expression
\[ J_\chi(f) = m_{\cusp}(\sigma) \int_{i\aaa_P^*} \trace\left(\MMM_P(\sigma_\lambda) \III_P(\sigma_\lambda, f)\right) \d \lambda. \]
We now briefly describe the terms involved and compute them for $G=\sl(2)$. Here $m_{\cusp}(\sigma)$ is the multiplicity of $\sigma$ in the representation $R_{M_P, \cusp}$. The operator $\MMM_P(\sigma_\lambda)$ is a smooth function in $\lambda \in i\aaa_P^*$ corresponding to a $(G, M)$-family of intertwining operators which we describe as follows. 

For $\sigma$ an irreducible representation of $M_0(\A)^1$ and $\lambda \in \aaa_{M, \C}^*, \III_P(\sigma_\lambda)$ denotes the induced representation
\[ \Ind_{P(\A)}^{G(\A)} \left( nm \mapsto \sigma(m) \exp(\sprod{\lambda}{H_M(m)}) \right). \]
The operator $\III_P(\sigma_\lambda, f)$ is obtained from this representation by integrating over $f$ in the usual manner. The space of this operator is $\HHH_P$ and in particular is independent of $\lambda \in \aaa_{M, \C}^*$. Given $Q \in \PPP(M)$ there is a unique $s \in W(\aaa_P, \aaa_Q)$ such that $s(P) = Q$. The operator
\[ M_{Q|P}(s, \lambda) = M(s, \lambda) = M_{Q|P}(\lambda): \HHH_P \to \HHH_Q \]
defined by
\[ (M(s, \lambda)\Phi)(x) = \int \Phi(w_s^{-1}nx) \exp(\sprod{\lambda + \rho_P}{H_P(w_x^{-1}nx)}) 
		\exp(-\sprod{s\lambda + \rho_Q}{H_Q(x)}) \d n \]
where the integral is taken over $N_Q(\A) \cap w_s N_P(\A) w_s^{-1} \bs N_Q(\A)$ intertwines the $G(\A)$-spaces $\HHH_P$ and $\HHH_Q$. It is a deep consequence of Langlands' theory of Eisenstein series that this integral converges absolutely for $\lambda$ in a ``positive'' cone in $i\aaa_M^*$ and has meromorphic continuation to $\aaa_{M, \C}^*$. These intertwining operators play a central role in the theory of automorphic forms. As $Q$ varies over $\PPP(M)$ or equivalently as $s$ varies over cosets of $W^G/W^M$, the family
\[ \{ M_{Q|P}(s, \lambda)^{-1} \circ M_{Q|P}(s, \lambda + \Lambda) : Q \in \PPP(M), \ \Lambda \in i\aaa_M^* \} \]
can be seen to be a $(G, M)$-family and consequently gives rise to a smooth function of $\lambda \in i\aaa_M^*$, namely
\[ \MMM_M(\sigma_\lambda) = \lim_{\Lambda \to 0} \ \sum_{Q \in \PPP(M)} \theta_Q(s\lambda)^{-1} M(\tilde w_s, \lambda)^{-1} M(\tilde w_s, \lambda + \Lambda). \]
Estimates on this operator $\MMM_M(\sigma_\lambda)$ have been used to prove the absolute convergence of the spectral side for $\gl(n)$ by M{\"u}ller-Speh \cite{MS} and general $G$ by Finis-Lapid-M{\"u}ller \cite{FLM}.

When $G = \sl(2)$ and $\chi \in \XXX$ corresponds to the pair $(G, \sigma)$ the spectral side
\[ J_\chi(f) = m_{\text{cusp}}(\sigma) \int_{i\aaa_P^*} \trace\left(\MMM_M(\sigma_\lambda) \III_P(\sigma_\lambda, f)\right) \d \lambda \]
reduces to $\trace \sigma(f)$. Since $P=G$ the integral vanishes and $\MMM_M(\sigma, \lambda) = 1$. Moreover $m_\text{cusp} (\sigma) = 1$ due to a result of Ramakrishna \cite{MR1792292}. Thus when $P=G$ we recover the expression for trace of $f$ as in the co-compact case namely
\[ J_\chi(f) = \trace \pi(f) \]
where $\chi \leftrightarrow (G, \pi)$. The classes $\chi = [(P_0, \sigma), (P_0, \sigma^{-1})]$ for $\sigma$ a character of $M_0(\Q) \bs M_0(\A)^1$ with $\sigma^2 \neq 1$ are also unramified since the long element takes $\sigma$ to $\sigma^{-1}$ and vice versa. The contribution of such classes to the spectral side is given by an integral namely,
\[ J_\chi(f) = \int_{i\aaa_0^*} \trace\left( \MMM_{M_0}(\sigma_\lambda) \III_{P_0}(\sigma_\lambda, f) \right) \d \lambda. \]
Note that the multiplicity $m_{\text{cusp}}(\sigma)$ is 1 by an application of the Peter-Weyl theorem since the group $M_0(\Q)\bs M_0(\A)^1$ is compact. 


Since the classes $(P_0, \sigma)$ are not unramified when $\sigma^2 = 1$, we need to analyze them using the fine spectral expansion that Arthur develops in \cite{MR681738}. The analysis of the spectral side is very technical and depends on a previous result on the Paley-Wiener theorem \cite{MR697608} so we will content ourselves with describing the final statement and the terms therein. 
% It makes use, among other things, of truncated Eisenstein series, a combinatorial analysis of various $(G, M)$-families and the existence of normalizing factors which normalize the intertwining operators. 
For $\chi \in \XXX$, the expression for $J_\chi(f)$ is given as a sum over standard Levi subgroups $M \subseteq L, \pi \in \Pi_{\text{unit}}(M(\A)^1), s \in W^L(M)_{\reg}$ of 
\begin{equation} \label{fine_spec} 
	\frac{\mod{W_0^M}}{\mod{W_0^G}} \left| \det \left(s - 1|_{\aaa_M^G} \right) \right|^{-1} 
		\int_{i\aaa_L^* / i\aaa_G^*} \trace \left( \MMM_L(P, \lambda) M_P(s, 0) \III_{P, \chi, \pi}(\lambda, f) \right) \d \lambda. 
\end{equation}

Let us understand the terms involved. The set $W^L(M)_{\reg}$ consists of elements $s \in W^L(M) = W^L(\aaa_M, \aaa_M)$ which satisfy $\det(s-1|_{\aaa_M^G}) \neq 0$. The operator $\MMM_L(P, \lambda)$ is the smooth function in $\lambda \in i\aaa_L^*$ corresponding to the $(G, L)$-family obtained by restricting the $(G, M)$-family 
\[ \{ \MMM_Q(\Lambda, \lambda, P) := M_{Q|P}(\lambda)^{-1} M_{Q|P}(\lambda + \Lambda) : Q \in \PPP(M), \Lambda \in i\aaa_M^* \} \]
of intertwining operators from $i\aaa_M^*$ to $i\aaa_L^*$. If $\lambda \in i\aaa_L^*$ and $s \in W^L(M)$ then $M_P(s, \lambda) = M_{P|P}(s, \lambda)$ is independent of $\lambda$, since $i\aaa_L^*$ is fixed by $s$. Arthur denotes this by $M_P(s, 0)$. Finally $\III_{P, \chi, \pi}(\lambda, f)$ is the restriction of the operator $\III_{P, \chi}(\lambda, f)$ on $\HHH_P$ to the subspace 
\[ \HHH_{P, \chi, \pi} = \HHH_{P, \chi} \cap \HHH_{P, \pi} \]
of $\HHH_{P, \chi}$. We have a decomposition of $\HHH_{P, \chi}$ as
\[ H_{P, \chi} = \bigoplus_\pi \HHH_{P, \chi, \pi} \]
where $\pi$ ranges over the set $\Pi_{\unit}(M_P(\A)^1)$ of irreducible unitary representations of $M_P(\A)^1$ and $\HHH_{P, \chi, \pi} = \HHH_{P, \chi} \cap \HHH_{P, \pi}$. Recall that $\HHH_{P, \chi}$ (resp. $\HHH_{P, \pi}$) consists of vectors $\Phi \in \HHH_P$ such that for almost every $x \in G(\A), \Phi_x \in \Ltwo_{\text{disc}}(M_P(\Q)\bs M_P(\A)^1)$ (resp. $\Phi_x \in \Ltwo(M_P(\Q)\bs M_P(\A)^1)$ is a matrix coefficient of $\pi$). 

Arthur's proof of the fine spectral expansion involves deep results that allow him to interchange two limits ($T \to \infty$ and $\varepsilon \to 0$). One ingredient in the proof is the existence of normalizing factors which when multiplied with intertwining operators, bestow nice properties to them (make symmetric in $P$ and $Q$, for instance). More precisely, the local normalized intertwining operators
\begin{equation} \label{rqp}
	R_{Q|P}(\pi_{v, \lambda}) := r_{Q|P}(\pi_{v, \lambda})^{-1} J_{Q|P}(\pi_{v, \lambda})
\end{equation}
satisfy
\[ R_{Q'|P}(\pi_{v, \lambda}) = R_{Q'|Q}(\pi_{v, \lambda}) R_{Q|P}(\pi_{v, \lambda}) \]
for any $P, Q, Q' \in \PPP(M)$. These normalizing factors $r_{Q|P}(\pi_{v, \lambda})$ are known to contain important arithmetic information and play an important role in the analysis of the spectral side. \cref{rqp} is a product $(G, M)$-family and for such families the corresponding smooth function has an explicit expression (see \cite{clay}*{\S 17}). For $\pi \in \Pi_{\text{unit}}(M(\A)^1)$, Arthur defines 
\begin{equation} \label{jmpi}
	J_M(\pi, f) = \int_{i\aaa_M^*} \trace \left( \RRR_M(\pi_\lambda, P) \III_P(\pi_\lambda, \tilde f) \right) \d \lambda,
\end{equation}
where $\tilde f$ is any $K$-finite function in $\CCC_c^\infty(G(\A))$ whose restriction to $G(\A)^1$ is $f$ and $\RRR_M(\pi_\lambda, P)$ is the smooth function corresponding to the $(G, M)$-family
\[ \{ R_{Q|P}(\pi_\lambda)^{-1} R_{Q|P}(\pi_{\lambda + \Lambda}), \qquad Q \in \PPP(M), \ \Lambda \in i\aaa_M^* \}. \]
Arthur introduces a measurable set of induced representations \cite{clay}*{Equation (22.6)}, namely
\[ \Pi_t(G) = \{ \pi_\lambda^G : M \in \LLL, \pi \in \Pi_{t, \text{disc}}(M), \lambda \in i\aaa_M^*/i\aaa_G^*  \} \]
equipped with a measure $\d \pi_\lambda^G$ such that 
\[ \int_{\Pi_t(G)} \phi(\pi_\lambda^G) \d \pi_\lambda^G = \sum_{M \in \LLL} \frac{\mod{W_0^M}}{\mod{W_0^G}} 
	\sum_{\pi \in \Pi_{t, \text{disc}}(M)} \int_{i\aaa_M^* / i\aaa_G^*} \phi(\pi_\lambda^G) \d \lambda \]
for functions $\phi$ on $\Pi_t(G)$. Here the suffix $t$ indicates those representations whose infinitesimal character has norm $t>0$. Notice the resemblance with \cref{fine_spec}. Setting
\[ a^G(\pi_\lambda^G) = a_{\text{disc}}^M(\pi) r_M^G(\pi_\lambda), \qquad M \in \LLL, \pi \in \Pi_{t, \text{disc}}(M), \lambda \in i\aaa_M^* / i\aaa_G^*, \]
and 
\[ \Pi(M)^T = \bigcup_{t > 0} \Pi_t(M), \qquad T > 0 \]
this formalism allows for a more elegant expression for the spectral side, namely
\[ J_{\text{spec}}(f) = \lim_{T \to \infty} \sum_{M \in \LLL} \frac{\mod{W_0^M}}{\mod{W_0^G}} \int_{\Pi(M)^T} a^M(\pi) J_M(\pi, f) \d \pi. \]

When Arthur developed this expression for the spectral side, it was not known whether the limit over $T \to \infty$ is finite. For the purpose of functoriality wherein information about the spectrum of two groups is obtained by comparing their geometric sides, one does not require this absolute convergence. However the trace formula can also be used in isolation (for instance, in Selberg's original work) for which one would need to know the spectral side converges absolutely. As we mentioned above, this is now known \cite{FLM} so the limit over $T$ above is redundant. 

In the case of $\sl(2)$ when $M$ is the minimal standard Levi subgroup $M_0$, the set $\Pi(M_0)$ is easily seen to be that of irreducible unitary representations of $M_0(\Q)\bs M_0(\A)^1$. Thus for $G = \sl(2)$, the spectral side is given by
\begin{equation} \label{spec}
	J_{\text{spec}}(f) = \sum_{\pi \in \Pi_{\text{cusp}}(G)} \trace \pi(f) + \frac{1}{2}
		\sum_{\pi \in \Pi(M_0(\Q)\bs M_0(\A)^1)} J_{M_0}(\pi, f). 
\end{equation}
where $J_{M_0}(\pi, f)$ is given by \cref{jmpi}. 

%%%%%%%%%%%%%%%%%%%%%%%%%%%%%%%%%%%%%%%%%%%%%%%%%%%%%%%%
%     The invariant trace formula
%%%%%%%%%%%%%%%%%%%%%%%%%%%%%%%%%%%%%%%%%%%%%%%%%%%%%%%%
\section{The invariant trace formula}

In sections 22 and 23 Arthur explains the need to have an invariant trace formula and the mechanics of modifying the fine geometric and spectral expansions to get terms which are invariant distributions. He does a minor modification of the coefficient $a^M(\gamma)$, see \cite{clay}*{Equation (22.2)} due to which the geometric side can be written as a limit over increasing set $S$ of places of
\[ \sum_{M \in \LLL} \frac{\mod{W_0^M}}{\mod{W_0^G}} \sum_{\gamma \in (M(\Q))_{M, S}} a^M(\gamma) J_M(\gamma, f). \]

When $G=\sl(2)$ and $M=M_0$, the representatives of $(M_0(\Q))_{M, S}$ can be chosen in the set 
\[ \left \{ \begin{pmatrix} t & 0 \\ 0 & t^{-1} \end{pmatrix} : t_p \in \Z_p \ \forall \ p \in S \right \}. \]
Thus the geometric expansion for $\sl(2)$ can be written more succinctly as  
\[ J_{\text{geom}}(f) = \sum_{\gamma \in (G(\Q))_{G, S}} a^G(\gamma) J_G(\gamma, f) + 
		\frac{1}{2} \sum_{\gamma \in M_0(\Z_S)} a^{M_0}(\gamma) J_{M_0}(\gamma, f). \]
Note that the term $a^G(\gamma) = \vol(G(\Q)_\gamma \bs G(\A)^1_\gamma)$ is independent of $S$ (always the case) whereas $a^{M_0}(\gamma) = \vol (M_0(\Z_S)\bs M_0(\Q_S)^1)$ is not. 

In \cref{sec:geom}, we defined a distribution $J$ on $G(\A)$ to be invariant if for every $y \in G(\A)$ and every $f \in \CCC_c^\infty(G(\A)), J(f) = J(f^y)$, where $f^y$ is conjugation with respect to $y$. However when we restrict $f$ to be in the Hecke algebra, that is, $f$ is bi-$K$-invariant then $f^y$ need not be so. We modify the definition of invariance as follows. A linear form $J$ on $\HHH(G)$ is said to be \textit{invariant} if for every $f, h \in \HHH(G)$, we have $J(f) = J(f*h) = J(h*f)$.

So far we have the spectral and geometric expansions for $\sl(2)$ as
\begin{equation} \label{j_spec}
	J_{\text{spec}}(f) = \sum_{\substack{\pi \text{cuspidal} \\ \text{rep. of } \gl(2)}} \trace \pi(f) + \frac{1}{2} \sum_{\pi \in \Pi(M_0(\Q) \bs M_0(\A)^1)} J_{M_0}(\pi, f), 
\end{equation}
\begin{equation} \label{j_geom}
	J_{\text{geom}}(f) = \sum_{\gamma \in (G(\Q)_{G, S})} a^G(\gamma) J_G(\gamma, f) + \frac{1}{2} \sum_{\gamma \in M_0(\Z_S)} a^{M_0}(\gamma) J_{M_0}(\gamma, f). 
\end{equation}
The first sum on both sides corresponding to $G$, is invariant whereas the second, corresponding to the Levi subgroup $M_0$ is not. The invariant trace formula for $\sl(2)$ is simpler than that for general $G$ because $M_0$ is abelian (so all unitary representations are tempered) and since $M_0$ is minimal and maximal (nontrivial) Levi subgroup. To make the above distributions invariant, we would like to do one of the following, namely for a fixed $\gamma \in M_0(\Z_S)$, subtract the spectral contribution, 
\begin{equation} \label{sub_spec}
J_{M_0}(\gamma, f) - \int_{\pi \in \Pi_{\text{unit}}(M_0(\Q_S)^1)} \pi(\gamma^{-1}) J_M(\pi, f) \d \pi, \qquad \gamma \in M_0(\Q_S)^1,
\end{equation}
or vice versa, that is, for fixed $\pi$ subtract the geometric contribution,
\[ J_{M_0}(\pi, f) - \int_{\gamma \in M_0(\Q_S)} \pi(\gamma) J_{M_0}(\gamma, f) \d \gamma, \qquad \pi \in \Pi_{\text{unit}}(M_0(\Q_S)). \]

In Section 22, Arthur provides justification that it is more reasonable to subtract the spectral contribution from $J_{M_0}(\gamma, f)$. Indeed he is able to show that an appropriate Poisson summation formula applied to the sum in \cref{j_spec} corresponding to $M_0$ would yield the expression in \cref{sub_spec}. For this he needs to define a Fourier transform on the space of continuous invariant forms on $\HHH(G)$ which are ``supported on characters''. 

For a general group $G$, the characters of tempered representations provide a map
\[ f \mapsto f_G(\pi) = \trace \pi(f) \]
from the Hecke algebra $\HHH(G)$ to the set $\III(G(\Q_S))$ of complex-valued functions on $\Pi_{\text{temp}}(G(\Q_S))$. Given a continuous invariant form $I$ on $\HHH(G(\Q_S))$, we say $I$ is supported on characters if $I(f) = 0$ for all $f \in \HHH(G(\Q_S))$ such that $f_G = 0$. This being the case, it is clear that there is a continuous linear form $\widehat I$ on $\III(G(\Q_S))$ such that 
\[ I(f) = \widehat I(f_G), \qquad \forall f \in \HHH(G(\Q_S)). \]
\[ \xymatrix{ \HHH(G(\Q_S)) \ar[rr]^{f \mapsto f_G} \ar[dr]_I && \III(G(\Q_S)) \ar@{-->}[dl]^{\widehat I} \\ & \C & }
\]
The form $\widehat I$ is known as the \underline{invariant Fourier transform} of $I$. It turns out that the image of the map
\[ f \mapsto f_M : \HHH(G(\Q_S)) \to \III(M(\Q_S)) \]
when $f_M(\pi) = J_M(\pi, f)$ does not lie in $\III(M(\Q_S))$. To deal with this, Arthur enlarges the space $\III(M(\Q_S))$ appropriately. Without going into the technical aspects of this, we remark that Arthur is able to show (under a mild hypothesis on the set $S$ of places) that the map $I \mapsto \widehat I$ extends to an isomorphism from the space $\HHH_{\text{ac}}(G(\Q_S))$ (which contains $\HHH(G(\Q_S))$), to the space of continuous linear forms on $\III_{\text{ac}}(G(\Q_S))$ (which contains $\III(G(\Q_S))$).

Here the map $f \mapsto f_G$ is given by
\[ f_G(\pi, Z) = \trace \pi(f^Z), \qquad \pi \in \Pi_{\text{temp}}(G(\Q_S)), \ Z \in \aaa_{G, S} \]
where $\aaa_{G, S}$ is a closed subgroup of $\aaa_G$ which equals $\aaa_G$ if $S$ contains the archimedean valuation, and $f^Z$ is the closed subset of $G(\Q_S)$ given by
\[ G(\Q_S)^Z = \{ x \in G(\Q_S) : H_G(x) = Z \}. \]
When $G = \sl(2)$, clearly $\aaa_{G, S} = 0$ and $G(\Q_S)^Z$ is nonempty if and only if $Z = 0$. 

The idea is to be able to define continuous linear maps from $\HHH_{\text{ac}}(G(\Q_S))$ to $\III_{\text{ac}}(M(\Q_S))$ where $M$ is a Levi subgroup of $G$. To such a subgroup $M, \pi \in \Pi(M(\Q_S)), \lambda \in \aaa_{M, \C}^*$ and $Z \in \aaa_{G, S}$, Arthur defines a meromorphic function
\[ J_M(\pi_\lambda, f^Z) := \trace \left( \RRR_M(\pi_\lambda, P) \III_P(\pi_\lambda, f^Z) \right) \]
and its natural extension for $X \in \aaa_{M, S}$ namely
\[ J_M(\pi, X, f) := \int_{i\aaa_M^* / i\aaa_G^*} J_M(\pi_\lambda, f^Z) \exp(-\sprod{\lambda}{X}) \d \lambda. \]
Note the formal resemblance with \cref{jmpi}. Using this Arthur defines the map
\[ \phi_M : \HHH_{\text{ac}}(G(\Q_S)) \to \III_{\text{ac}}(M(\Q_S)) \]
where $\phi_M(f)(\pi, X) = \phi_M(f, \pi, X) = J_M(\pi, X, f)$ with $\pi \in \Pi_{\text{temp}}(M(\Q_S))$ and $X \in \aaa_{M, S}$. In \cite{MR999488}, Arthur shows that $\phi_M$ is a continuous linear transform. We now state a theorem about the existence of the aforementioned Fourier transform $I \mapsto \widehat I$ and the invariant trace formula defined in terms of the invariant distributions $I_M(\gamma, f)$ and $I_M(\pi, f)$. 

\begin{theorem} \label{inv_geom}
For $\gamma \in M(\Q_S)$ and $f \in \HHH_{\text{ac}}(G(\Q_S))$ there are invariant linear forms $I_M(\gamma, f)$ that are supported on characters (meaning $\widehat I_M(\circ, f)$ exists), and which satisfy 
\[ I_M(\gamma, f) = J_M(\gamma, f) - \sum_{\substack{L \in \LLL(M) \\ L \neq G}} \widehat I_M^L(\gamma, \phi_L(f)). \]
\end{theorem}

\begin{theorem} \label{inv_spec}
For $\pi \in \Pi(M(\Q_S)), X \in \aaa_{M, S}$ and $f \in \HHH_{\text{ac}}(G(\Q_S))$, there are invariant linear forms $I_M(\pi, X, f)$ supported on characters which satisfy
\[ I_M(\pi, X, f) = J_M(\gamma, f) - \sum_{\substack{L \in \LLL(M) \\ L \neq G}} \widehat I_M^L(\gamma, \phi_L(f)). \]
\end{theorem}

\begin{theorem} \label{invtf}
For any $f \in \HHH(G)$, the invariant trace formula is given by the equality of the geometric expansion
\[ I_{\text{geom}}(f) = \lim_S \sum_{M \in \LLL} \frac{\mod{W_0^M}}{\mod{W_0^G}} \sum_{\gamma \in \Gamma(M)_S} a^M(\gamma) I_M(\gamma, f) \]
with the spectral expansion
\[ I_{\text{spec}}(f) = \lim_T \sum_{M \in \LLL} \frac{\mod{W_0^M}}{\mod{W_0^G}} \int_{\Pi(M)^T} a^M(\pi) I_M(\pi, f) \d \pi. \]
\end{theorem}

The three theorems above are proven simultaneously by induction on the rank of $G$, a theme which is frequently seen in Arthur's work. It is clear from \cref{inv_geom,inv_spec} that when $M=G$, then
\[ I_G(\gamma, f) = J_G(\gamma, f) = \int_{G_\gamma(\Q_S)\bs G(\Q_S)} f(x^{-1}\gamma x) \d x \]
and 
\[ I_G(\pi, f) = J_G(\pi, f) = \trace \pi(f). \]
Moreover when $G = \sl(2)$ and $M = M_0$ and $\gamma \in M_0(\Z_S)$ then 
\[ I_{M_0}(\gamma, f) = J_{M_0}(\gamma, f) - \widehat I_{M_0}^{M_0} (\gamma, \phi_{M_0}(f)). \]
Consider the map
\[ \xymatrix{\HHH_{\text{ac}}(M_0(\Q_S)) \ar[rr]^{\phi_{M_0}}  \ar[dr]_{I_{M_0}^{M_0}(\gamma, \circ)} && \III_{\text{ac}}(M_0(\Q_S)) \ar@{-->}[dl]^{\widehat I_{M_0}^{M_0} (\gamma, \circ)} \\ & \C & } \]
Since $I_{M_0}^{M_0}(\gamma, \circ)$ is supported on characters, 
\[ \widehat I_{M_0}^{M_0}(\gamma, \phi_{M_0}(f)) = I_{M_0}^{M_0}(\gamma, f) = f(\gamma). \]
Thus, $I_{M_0}(\gamma, f) = J_{M_0}(\gamma, f) - f(\gamma)$. Indeed as Arthur explains in Remark 1 following Theorem 23.2, an application of the Fourier inversion theorem to the abelian group $M_0(\Q_S)^1$ gives
\begin{equation} \label{f_gamma}
f(\gamma) = \int_{\widehat{M_0(\Q_S)^1}} \pi(\gamma) \widehat f(\pi) \d \pi, 
\end{equation}
which equals the second term in \cite{clay}*{Equation (22.13)}. 

On the other hand, 
\[ I_G(\pi, X, f) = J_G(\pi, X, f) = J_G(\pi, 0, f) = \trace \pi(f). \]
When $M = M_0$, 
\[ I_{M_0}(\pi, X, f) = J_{M_0}(\pi, X, f) - \widehat I_{M_0}^{M_0}(\pi, X, \phi_{M_0}(f)) \]
and
\[ \widehat I_{M_0}^{M_0}(\pi, X, \phi_{M_0}(f)) = I_{M_0}^{M_0}(\pi, X, f) = J_{M_0}^{M_0}(\pi, 0, f) \]
so $I_{M_0}(\pi, X, f) = 0$. 

Thus the invariant trace formula for $\sl(2)$ is given by
\begin{subequations}
\begin{empheq} [box=\widefbox]{align}
 I_{\text{geom}}(f) = \lim_S \sum_{\gamma \in \Gamma(G)_S} a^G(\gamma) & J_G(\gamma, f) + \frac{1}{2} \sum_{\gamma \in \Gamma(M_0)_S} \left[ a^{M_0}(\gamma) J_{M_0}(\gamma, f) - f(\gamma) \right], \\
 I_{\text{spec}}(f) & = \sum_{\substack{\pi \text{ cusp} \\ \text{ form of } G}} \trace \pi(f). 
\end{empheq}
\end{subequations}
Although it might seem counterintuitive at first glance that the `extra' term to be subtracted from $J_{M_0}(\gamma, f)$ is just $f(\gamma)$, this indeed is the spectral contribution of the Levi subgroup $M_0$ through \cref{f_gamma}.



%%%%%%%%%%%%%%%%%%%%%%%%%%%%%%%%%%%%%%%%%%%%%%%%%%%%%%%%
%     Absolute convergence and relation with Beyond Endoscopy
%%%%%%%%%%%%%%%%%%%%%%%%%%%%%%%%%%%%%%%%%%%%%%%%%%%%%%%%
\section{Absolute convergence and relation with `Beyond Endoscopy'}

Suppose $G$ is a connected reductive group and assume for simplicity it is defined over $\Z$. Let
\[ r : {}^LG \to \gl(V) \]
be a representation of its dual group. For every finite prime $p$ there exists a unique function $\varphi_{p, r, s}$ defined on $G(\Q_p)$ which is invariant on the left and right under $G(\Z_p)$ and which satisfies
\[ \trace \pi_p(\varphi_{p, r, s}) = L_p(s, \pi_p, r) \]
for $\Re(s) \gg 0$ and for every irreducible admissible representation $\pi_p$ of $G(\Q_p)$. This is an application of the Satake isomorphism (see \cite{MR3220933}). At the archimedean place let $\varphi_\infty$ be a smooth function on $G(\R)$ which belongs, along with its derivatives in $\Lone(G(\R))$. We form the test function
\[ \varphi_{r, s}(g) = \varphi_\infty(g_\infty) \displaystyle\prod_p \varphi_{p, r, s}(g_p). \]
This function is not of compact support so cannot be directly applied to the trace formula. However the absolute convergence of the trace formula for such functions has been proven, for the spectral side in \cite{FLM} and the geometric side in \cite{FL16}. A corresponding extension to the twisted trace formula is the work of the author \cite{Par19}. If this function is considered as a test function to the trace formula, the contribution of any discrete automorphic representation of any $\pi = \otimes_v \pi_v$ to the spectral side of the trace formula would be nonzero if and only if $\pi$ is unramified outside $\infty$ and in this case equal to $m(\pi) \trace \pi_\infty(\varphi_\infty) L^\infty(s, \pi^\infty, r)$. The broad picture of `Beyond Endoscopy' is to define $L$-functions imitating the method of Godement-Jacquet and weigh the trace formula by $L$-functions and then compare $r$-trace formulas on different groups. 

%%%%%%%%%%%%%%%%%%%%%%%%%%%%%%%%%%%%%%%%%%%%%%%%%%%%%%%%
%     	Arthur notes errata
%%%%%%%%%%%%%%%%%%%%%%%%%%%%%%%%%%%%%%%%%%%%%%%%%%%%%%%%
\textbf{Note}: A list of typographical errors in the Clay notes can be found at \url{https://github.com/aaparab/nus_langlands/blob/master/arthur_clay_typos.pdf}. For any changes or suggestions please contact the author via email or create an issue on the Github webpage. 

%%%%%%%%%%%%%%%%%%%%%%%%%%%%%%%%%%%%%%%%%%%%%%%%%%%%%%%%
%     Bibliography
%%%%%%%%%%%%%%%%%%%%%%%%%%%%%%%%%%%%%%%%%%%%%%%%%%%%%%%%

\begin{bibdiv}
\begin{biblist}*{labels={alphabetic}}

    \bib{duke}{article} % Arthur - A trace fromula I (1978)
    {
       author={Arthur, James G.},
       title={A trace formula for reductive groups. I. Terms associated to
       classes in $G({\bf Q})$},
       journal={Duke Math. J.},
       volume={45},
       date={1978},
       number={4},
       pages={911--952},
       issn={0012-7094},
       review={\MR{518111}},
    }

	\bib{MR625344}{article}
	{
       author={Arthur, James},
       title={The trace formula in invariant form},
       journal={Ann. of Math. (2)},
       volume={114},
       date={1981},
       number={1},
       pages={1--74},
       issn={0003-486X},
       review={\MR{625344}},
       doi={10.2307/1971376},
    }
    
    \bib{MR681738}{article}{
       author={Arthur, James},
       title={On a family of distributions obtained from Eisenstein series. II.
       Explicit formulas},
       journal={Amer. J. Math.},
       volume={104},
       date={1982},
       number={6},
       pages={1289--1336},
       issn={0002-9327},
       review={\MR{681738}},
       doi={10.2307/2374062},
    }
    
    \bib{MR697608}{article}{
       author={Arthur, James},
       title={A Paley-Wiener theorem for real reductive groups},
       journal={Acta Math.},
       volume={150},
       date={1983},
       number={1-2},
       pages={1--89},
       issn={0001-5962},
       review={\MR{697608}},
       doi={10.1007/BF02392967},
    }
    
	\bib{MR828844}{article}{
       author={Arthur, James},
       title={A measure on the unipotent variety},
       journal={Canad. J. Math.},
       volume={37},
       date={1985},
       number={6},
       pages={1237--1274},
       issn={0008-414X},
       review={\MR{828844}},
       doi={10.4153/CJM-1985-067-0},
    }

    \bib{MR835041}{article}{
       author={Arthur, James},
       title={On a family of distributions obtained from orbits},
       journal={Canad. J. Math.},
       volume={38},
       date={1986},
       number={1},
       pages={179--214},
       issn={0008-414X},
       review={\MR{835041}},
       doi={10.4153/CJM-1986-009-4},
    }
    
	\bib{MR932848}{article}{
       author={Arthur, James},
       title={The local behaviour of weighted orbital integrals},
       journal={Duke Math. J.},
       volume={56},
       date={1988},
       number={2},
       pages={223--293},
       issn={0012-7094},
       review={\MR{932848}},
       doi={10.1215/S0012-7094-88-05612-8},
    }
    
    \bib{MR999488}{article}{
       author={Arthur, James},
       title={Intertwining operators and residues. I. Weighted characters},
       journal={J. Funct. Anal.},
       volume={84},
       date={1989},
       number={1},
       pages={19--84},
       issn={0022-1236},
       review={\MR{999488}},
       doi={10.1016/0022-1236(89)90110-9},
    }

    \bib{clay}{article}{		% Clay notes
       author={Arthur, James},
       title={An introduction to the trace formula},
       conference={
          title={Harmonic analysis, the trace formula, and Shimura varieties},
       },
       book={
          series={Clay Math. Proc.},
          volume={4},
          publisher={Amer. Math. Soc., Providence, RI},
       },
       date={2005},
       pages={1--263},
       review={\MR{2192011}},
    }

	\bib{MR546598}{article} % Corvallis -- Borel & Jacquet 
    {
       author={Borel, A.},
       author={Jacquet, H.},
       title={Automorphic forms and automorphic representations},
       note={With a supplement ``On the notion of an automorphic
       representation''\ by R. P. Langlands},
       conference={
          title={Automorphic forms, representations and $L$-functions},
          address={Proc. Sympos. Pure Math., Oregon State Univ., Corvallis,
          Ore.},
          date={1977},
       },
       book={
          series={Proc. Sympos. Pure Math., XXXIII},
          publisher={Amer. Math. Soc., Providence, R.I.},
       },
       date={1979},
       pages={189--207},
       review={\MR{546598}},
    }
    
    \bib{FL11b}{article} %FL11 Combinatorial setup
    {
       author={Finis, Tobias},
       author={Lapid, Erez},
       title={On the spectral side of Arthur's trace formula---combinatorial setup},
       journal={Ann. of Math. (2)},
       volume={174},
       date={2011},
       number={1},
       pages={197--223},
       issn={0003-486X},
       review={\MR{2811598}},
       doi={10.4007/annals.2011.174.1.6},
    }

   \bib{FL16}{article} %Geometric side FINAL
   {
	   author={Finis, Tobias},
	   author={Lapid, Erez},
	   title={On the continuity of the geometric side of the trace formula},
	   journal={Acta Math. Vietnam.},
	   volume={41},
	   date={2016},
	   number={3},
	   pages={425--455},
	   issn={0251-4184},
	   review={\MR{3534542}},
	   doi={10.1007/s40306-016-0176-x},
	}

    \bib{FLM}{article} % FLM .. 
    {
       author={Finis, Tobias},
       author={Lapid, Erez},
       author={M{\"u}ller, Werner},
       title={On the spectral side of Arthur's trace formula---absolute
       convergence},
       journal={Ann. of Math. (2)},
       volume={174},
       date={2011},
       number={1},
       pages={173--195},
       issn={0003-486X},
       review={\MR{2811597}},
       doi={10.4007/annals.2011.174.1.5},
    }
    
    \bib{MR0401654}{book}{
       author={Jacquet, H.},
       author={Langlands, R. P.},
       title={Automorphic forms on ${\rm GL}(2)$},
       series={Lecture Notes in Mathematics, Vol. 114},
       publisher={Springer-Verlag, Berlin-New York},
       date={1970},
       pages={vii+548},
       review={\MR{0401654}},
    }

    \bib{MR2058622}{article}{
       author={Langlands, Robert P.},
       title={Beyond endoscopy},
       conference={
          title={Contributions to automorphic forms, geometry, and number
          theory},
       },
       book={
          publisher={Johns Hopkins Univ. Press, Baltimore, MD},
       },
       date={2004},
       pages={611--697},
       review={\MR{2058622}},
    }
        
    \bib{MW}{book} %   Moeglin Waldspurger
    {
       author={M{\oe}glin, C.},
       author={Waldspurger, J.-L.},
       title={Spectral decomposition and Eisenstein series},
       series={Cambridge Tracts in Mathematics},
       volume={113},
       note={Une paraphrase de l'\'Ecriture [A paraphrase of Scripture]},
       publisher={Cambridge University Press, Cambridge},
       date={1995},
       pages={xxviii+338},
       isbn={0-521-41893-3},
       review={\MR{1361168}},
       doi={10.1017/CBO9780511470905}
    }
    
    \bib{MS}{article}		% Muller Speh
    {
       author={M\"uller, W.},
       author={Speh, B.},
       title={Absolute convergence of the spectral side of the Arthur trace
       formula for ${\rm GL}_n$},
       note={With an appendix by E. M.\ Lapid},
       journal={Geom. Funct. Anal.},
       volume={14},
       date={2004},
       number={1},
       pages={58--93},
       issn={1016-443X},
       review={\MR{2053600}},
       doi={10.1007/s00039-004-0452-0},
    }
	    
   \bib{MR3220933}{article}		% Ngo, PS article
   {
       author={Ng\^o, Bao Ch\^au},
       title={On a certain sum of automorphic $L$-functions},
       conference={
       title={Automorphic forms and related geometry: assessing the legacy of
          I. I. Piatetski-Shapiro},
          },
       book={
          series={Contemp. Math.},
          volume={614},
          publisher={Amer. Math. Soc., Providence, RI},
       },
       date={2014},
       pages={337--343},
       review={\MR{3220933}},
       doi={10.1090/conm/614/12270},
    }

    \bib{Par19}{article}{
       author={Parab, Abhishek},
       title={Absolute convergence of the twisted trace formula},
       journal={Math. Z.},
       volume={292},
       date={2019},
       number={1-2},
       pages={529--567},
       issn={0025-5874},
       review={\MR{3968914}},
       doi={10.1007/s00209-019-02290-0},
    }
    
    \bib{MR1792292}{article}{
       author={Ramakrishnan, Dinakar},
       title={Modularity of the Rankin-Selberg $L$-series, and multiplicity one
       for ${\rm SL}(2)$},
       journal={Ann. of Math. (2)},
       volume={152},
       date={2000},
       number={1},
       pages={45--111},
       issn={0003-486X},
       review={\MR{1792292}},
       doi={10.2307/2661379},
    }

    \bib{MR1068677}{article}{
       author={Stern, Leonid},
       title={On the equality of norm groups of global fields},
       journal={J. Number Theory},
       volume={36},
       date={1990},
       number={1},
       pages={108--126},
       issn={0022-314X},
       review={\MR{1068677}},
       doi={10.1016/0022-314X(90)90009-G},
    }
    
\end{biblist}
\end{bibdiv}

\end{document}  
