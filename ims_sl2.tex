%%%%%%%%%%%%%%%%%%%%%%%%%%%%%%%%%%%%%%%%%%%%%%%%%%%%%%%%%%%%%%%%%%%%%%%%%%%
%% ims9x6-ch1.tex - template file with chapter content text    17/7/07
%% Review Volume Title: Lecture Note Series
%%                      Institute for Mathematical Sciences, (IMS)
%%                      National University of Singapore     (NUS)
%%
%% The content, structure, format and layout of this style file
%% is the property of World Scientific Publishing Co. Pte. Ltd.
%% Copyright 1995, 2002 by World Scientific Publishing Co.
%% All rights are reserved.
%%
%% Trim Size: 9in x 6in
%% Text Area: 7.35in (include runningheads) x 4.5in
%% Main Text is 10/13pt
%%%%%%%%%%%%%%%%%%%%%%%%%%%%%%%%%%%%%%%%%%%%%%%%%%%%%%%%%%%%%%%%%%%%%%%%%%%

%\documentclass[draft]{ims9x6}
\documentclass{ims9x6}

\makeindex

\begin{document}

\chapter{DETECTING AND REVOKING COMPROMISED KEYS}

\markboth{T. Matsumoto}{Detecting and Revoking Compromised Keys}

\author{Tsutomu Matsumoto}

\address{Graduate School of Environment and Information Sciences\\
Yokohama National University\\
79-7, Tokiwadai, Hodogaya, Yokohama, 240-8501, Japan\\
tsutomu@mlab.jks.ynu.ac.jp}

\begin{abstract}
This note describes two correlated topics in cryptography. The first
topic is the entity\index{entity} exclusion, or how to distribute a
cryptographic\index{cryptographic}
key over a broadcasting shared by $n$ entities so that all but $d$
excluded entities can get a group\index{group} key.
In a system such as Pay-TV, Internet multicasting and group
mobile telecommunication, a dishonest
user or an unauthorized terminal should be excluded as quickly as
possible. We discuss the points of evaluation and history of the field
followed by concrete schemes smartly enabling the entity
exclusion. The second topic is how to discover the existence of a
``clone," that is, another entity with the same ID and the same secret
key as the original. This problem is rather hard to solve in general.
However, depending on environmental\index{environmental}
conditions there are approaches for solving the problem.
We suggest some effective ways for the clone discovery.
\end{abstract}

\vspace*{12pt}

\chaptercontents  %type this to generate out the chapter content text

\section{Introduction}
Imagine a system consisting of a lot of entities. An entity is what
creates, sends, receives, processes, modifies, or uses data.\index{data}
The system employs \hbox{cryptography} to maintain the
confidentiality, integrity, or authenticity of data\index{data}
exchanged among the entities.\index{entities}

A typical scenario\index{scenario} is that each entity is assigned a
unique identifier (ID) and an individualized private key as well as
other parameters. Security attained by cryptography is based on the
assumption that every private\index{private} key is kept
properly. Therefore two natural questions arise.

How can the system know that the assumption is satisfied?  How can the
system\index{system} revoke a particular key if there is a need to do
so?  Revoking a key may be restated as excluding an entity possessing
the corresponding key.\index{key}

\section{Entity Exclusion in Group Communication
Over a Broadcast Channel to All Users}
\subsection{Theme}
A {\em broadcast encryption\/} allows a distributor to send the same
message \hbox{simultaneously} over a broadcast channel\index{channel} to all
authorized users with confidentiality. Pay-TV via cable\index{cable}
and satellite networks,\index{networks!satellite}
Internet multicasts, and mobile\index{mobile}
group telecommunication such as a private mobile radio\index{radio}
or a taxi radio, are typical examples.  A secure and fast method to
distribute a shared key\index{key} (which is called a {\em group key}
in this note) to all the proper users is required.

The main part of this note focuses\index{focuses} on
the {\em entity exclusion}, or how to transmit a group key over
a broadcast channel\index{channel}
shared by $n$ entities so that all but $d$ excluded entities
can get the group key. For example, entity exclusion can prevent a
lost or stolen mobile terminal\index{terminal} to be used to
eavesdrop the secret broadcasting.\index{broadcasting} Entity
exclusion can also prevent unauthorized access to Pay-TV and Internet.

\subsection{Development}
A simple method of entity exclusion is that a distributor distributes
a new group key to each entity except the excluded users, in encrypted
form by a secret\index{secret} key of each user.  This method requires
each entity to keep only one secret key, while the distributor should
transmit $n-d$ encrypted\index{encrypted} new group keys.

Another simple method is that each entity has common keys for all
subsets of $n$ users.\index{users} This method does not require the
distributor to transmit any message, while each entity should keep a
lot of keys.

To improve this trade-off between the amount of transmission and the
key storage of each user, many
ideas~\cite{kn:AMM4,kn:B,kn:BMS1,kn:BMS2,kn:CGIM,kn:CMN,kn:FN,kn:LS,kn:MA,kn:MNTO,kn:WHA,kn:WGL}
have been proposed. Criteria\index{criteria} for evaluation
may be listed as
\begin{description}
\item {\bf flexibility} a fixed and privileged distributor is required
or not.
\item {\bf reusability} a secret key of each entity can be reused or
not.
\item {\bf scalability in complexity} the amount of transmission and
key storage of each entity is independent of the group scale $n$ or
not.
\end{description}

Berkovits~\cite{kn:B} proposed a scheme using secret sharing. Mambo,
Nishikawa, Tsujii and Okamoto~\cite{kn:MNTO} proposed a broadcast
communication scheme, which is efficient in terms of computation of
each entity and the amount of transmission. These two works can be
applied to key distribution for a pre-determined privileged subset,
but not to a dynamically changing subset.\index{subset}

A major step in key distribution with entity exclusion was marked when
Fiat and Naor proposed a scheme~\cite{kn:FN}. The scheme is resilient
to any coalition\index{coalition} of $d$ users, by extending an
original scheme, which
excludes a single user, using multi-layered hashing techniques. In the
scheme, each entity stores $O(d\log d\log n)$ keys and the distributor
broadcasts $O(d^2(\log d)^2\log n)$ messages. Blundo, Mattos and
Stinson extend this basic work in the papers~\cite{kn:BMS1,kn:BMS2},
and Luby and Staddon studied the trade-off in the paper~\cite{kn:LS}.

A second major step is a hierarchical key distribution scheme (called
HKDS in the current note) using a balanced binary tree. This was done
by two research\index{research} groups of Wallner, Harder and
Agee~\cite{kn:WHA} and Wong, Gouda and Lam~\cite{kn:WGL}. In this
scheme, the amount of transmission is $O((degree-1)\times\log n)$ and
the number of keys for each entity is $O(\log n)$, where $n$ is the
number of users on the broadcast channel\index{channel} and $degree$
is the number of users in the bottom subgroup of the binary
tree. Canetti, Garay, Itkis, Micciancio, Naor and Pinkas proposed an
extended method that reduces the amount of transmission in the
paper~\cite{kn:CGIM}. Canetti, Malkin and Nissim studied the trade-off
in the paper~\cite{kn:CMN}.

Recently, two works on the entity exclusion problem have been
presented: one by Kumar, Rajagopalan and Sahai~\cite{kn:KRS} and one
by Matsuzaki and Anzai~\cite{kn:MA}. In the Kumar-Rajagopalan-Sahai
scheme~\cite{kn:KRS} using algebraic geometric\index{geometric} codes,
each entity has an individual subset of a key-set. Redundant pieces of
message using an error-correcting code are encrypted by keys belonging
to users who are not excluded and are broadcast. The amount of
transmission is $O(d^2)$ regardless of $n$ and the key
storage\index{storage} of each entity is $O(d\times\log
n)$. Consequently, the scheme enables an efficient entity exclusion of
which the amount of transmission does not depend on the group
scale.\index{scale!group} However, this scheme still requires the key
storage that depends on the group scale $n$, and a fixed and
privileged distributor. \hbox{Matsuzaki} and Anzai proposed a
scheme~\cite{kn:MA} using mathematical techniques which are well-known
as {\em RSA common modulus attack\/} and {\em RSA low exponent
attack}. The Matsuzaki-Anzai scheme can simultaneously exclude up to
$d$ users. The amount of transmission is $O(d)$ regardless of $n$ and
each entity has only one key. Therefore, the scheme enables an
efficient entity exclusion when the group scale $n$ becomes large,
while the distributor\index{distributor} should pre-send a
secret\index{secret} key of each entity for every key distribution,
since the secret key cannot be reused, and the scheme requires a fixed
and privileged distributor who knows all secret keys of
users.\index{users}

Anzai, Matsuzaki and Matsumoto~\cite{kn:AMM4} proposed a scheme,
called {\bf MaSK}, which is the abbreviation of {\bf Ma}sked {\bf
S}haring of Group {\bf K}eys. They applied the {\em threshold
cryptosystems} given in~\cite{kn:DF} to achieve good reusability and
flexibility and scalability in complexity. Therefore the trick they
used may be interesting. The following two sections precisely
introduce them.

\section{MaSK: An Entity Exclusion Scheme}
We describe MaSK, the Anzai-Matsuzaki-Matsumoto scheme~\cite{kn:AMM4}.
MaSK can be based on an appropriate {\em Diffie-Hellman
Problem}~\cite{kn:MOV} defined over a finite cyclic group, including a
subgroup of Jacobian of an elliptic curve and so on. We describe it
over a prime field ${\bf Z}_p$. MaSK contains two phases: system
setup phase and key distribution\index{distribution!setup!key}
phase. Before explaining them we
should clarify the target\index{target} system and assumptions.

\subsection{Target system and assumptions}
The target system consists of the following:
\begin{description}
\item {\bf System manager:} A trusted party which decides system
parameters and sets each user's secret key. It manages a public
bulletin board.
\item {\bf Public bulletin board:} A public bulletin board keeps
system parameters and public keys for all users.
\item {\bf User \boldmath${i}$:} An entity labeled $i$ as its ID
number is a member of the group. We assume the number of total users
is $n$. Let $\Phi=\{1,2,\ldots,n\}$ be the set of all users.
\item {\bf Coordinator \boldmath${x}$:} A coordinator decides one or
more excluded user(s) and coordinates a group key distribution with
entity exclusion. We use the term ``coordinator" to distinguish it
from the fixed and privileged distributor discussed before. In the
scheme, any entity can become the coordinator.
\item {\bf Excluded user \boldmath$j$:} A user to be excluded by the
coordinator. Let $\Lambda(\subset\Phi)$ be a set of excluded users,
having $d$ users.
\item {\bf Valid user \boldmath$v$:} A user who is not an excluded
user. The set of all valid users forms the {\em group}.
\end{description}

In the target system, we make the following system assumptions:
\begin{arabiclist}[(2)]
\item[(1)] All users trust the system manager. The system manager
does not do anything illegal.
\item[(2)] All users have simultaneous access to the data that the
coordinator broadcasts.
\item[(3)] All users can get any data from the public bulletin board
at any time.
\item[(4)] The broadcast channel is not secure, i.e., anyone can see
the data on the broadcast channel.
\end{arabiclist}
and the following security assumptions:
\begin{arabiclist}[(2)]
\item[(1)] The {\em Diffie-Hellman Problem\/} is computationally hard
to solve.
\item[(2)] In $(k, n+k-1)$ threshold cryptosystems, anyone with less
than $k$ shadows cannot get any information about the secret $S$.
\item[(3)] Excluded users may conspire to get a group key.
\item[(4)] Excluded users may publish their secret information to
damage the system security.
\item[(5)] Valid users do not conspire with excluded users. If this
assumption is not satisfied, excluded users can get the group key from
valid users.
\item[(6)] Valid users do not publish their secret keys.
\item[(7)] The system manager manages the public bulletin board
strictly so as not to change it. Or, the public bulletin board has
each parameter with the certificate checked before using the public
parameters.
\end{arabiclist}

\subsection{System setup phase}
At the beginning, a system setup phase is carried out only once.
\begin{arabiclist}[(2)]
\item[(1)] A system manager decides a parameter $k$ satisfying
\[ 0\leq d\leq k-2<n, \]
where $n$ is the number of users in the group and $d$ is the upper
bound of the number of excluded users.
\item[(2)] The system manager decides the following system parameters
and publishes them to the public bulletin board:
\begin{itemlist}[$\bullet$]
\item[$\bullet$] $p$: a large prime number
\item[$\bullet$] $q$: a large prime number such that $q\mid p-1$ and
$n+k-1<q$
\item[$\bullet$] $g$: a $q^{th}$ root of unity over ${\bf Z}_p$
\item[$\bullet$] {\it sign}$(s,m)$: a secure signature generation
function, which outputs\index{outputs} a signature $Z$ of the message
$m$ using a secret key $s$. We use the signature scheme based on
DLP here, such as
DSA~\cite{kn:DSA} or Nyberg-Rueppel message recovery signature
scheme~\cite{kn:NR}.
\item[$\bullet$] {\it verify}$(y,Z)$: a secure signature verification
function, which checks the validity of the signature $Z$ using a
public key $y$. The function outputs the original message $m$ if the
signature $Z$ is ``valid''.
\end{itemlist}
\item[(3)] The system manager generates a system secret key
$S\in {\bf Z}_q$ and stores it secretly. And, the system manager
divides the system secret key $S$ into $n+k-1$ shadows
with the threshold $k$, using the well-known Shamir's secret
sharing scheme~\cite{kn:S} as follows:
\begin{alphlist}[(b)]
\item[(a)] The system manager puts $a_0=S$.
\item[(b)] The system manager defines the following equation over ${\bf
Z}_q$:
\begin{equation}	%(3.1)
f(x)=\sum_{f=0}^{k-1} a_fx^f\bmod q
\end{equation}
where $a_1,a_2,\ldots,a_{k-1}$ are random integers that satisfy the
following conditions:
\[
0\leq a_i\leq q-1\ \ {\rm for\ all\ } 1\leq i\leq k-1\ \
{\rm and }\ a_{k-1}\neq 0\, .
\]
\item[(c)] The system manager generates $n+k-1$ shadows as follows:
\begin{equation}	%(3.2)
s_i=f(i)\quad (1\leq i\leq n+k-1)
\end{equation}
\end{alphlist}
\item[(4)] The system manager distributes the shadows $s_1,\ldots,s_n$
to each user $1,\ldots,n$, respectively, in a secure manner. Each user
keeps its own shadow as its secret key.
\item[(5)] The system manager calculates public keys
$y_1,\ldots,y_{n+k-1}$ by the following equation:
\begin{equation}	%(3.3)
y_i=g^{s_i}\bmod p\quad (1\leq i\leq n+k-1)
\label{eq:PK}
\end{equation}
Then, the system manager publishes $y_1,\ldots,y_n$ on the public
bulletin board with the corresponding user's ID numbers. The remaining
$y_{n+1},\ldots,y_{n+k-1}$ and the corresponding ID numbers are also
published to the public bulletin board\index{board} as spare public keys. The
system manager may remove the secret keys $s_1,\ldots,s_{n+k-1}$ after
the system setup phase. The remaining tasks of the system manager are
to maintain the public bulletin\index{bulletin} board
and to generate a secret key and a public key of a new user.
\end{arabiclist}

\subsection{Key distribution phase}
\subsubsection{Broadcasting by a coordinator}
First, a coordinator $x$ generates a broadcast data $B(\Lambda, r)$ as
follows:
\begin{arabiclist}[(2)]
\item[(1)] The coordinator $x$ decides excluded users. Let $\Lambda$
be the set of excluded users and $d$ is the number of the excluded
users.
\item[(2)] The coordinator $x$ chooses $r\in Z_q$ at random and
picks $k-d-1$ integers from the set $\{n+1,\ldots,n+k-1\}$ and let
$\Theta$ be the set of chosen integers.\index{integers} Then, the coordinator
calculates $k-1$ exclusion data as follows:
\begin{equation}	%(3.4)
M_j=y_j^r\bmod p\quad (j\in\Lambda\cup\Theta)
\label{eq:RD}
\end{equation}
using the public keys of excluded users and the spare public keys on
the public bulletin board.
\item[(3)] The coordinator $x$ calculates the following preparation
data:
\begin{equation}	%(3.5)
X=g^r\bmod p
\end{equation}
\item[(4)] Using its own secret key $s_x$, the coordinator $x$
generates the signature for the data consisting of the preparation
data, own ID number, $k-1$ exclusion data, and the corresponding ID
numbers:
\begin{equation}	%(3.6)
Z=\hbox{\it sign}(s_x,X\|x\|\{[j,M_{j}]\mid j\in\Lambda\cup\Theta\})
\end{equation}
where $\|$ indicates {\em concatenation} of data.
\item[(5)] The coordinator $x$ broadcasts the following broadcast data
to all users:
\begin{equation}	%(3.7)
B(\Lambda,r)=Z\|x
\label{eq:BD}
\end{equation}
\end{arabiclist}

Next, the coordinator $x$ calculates a group key $U$ using its own
secret key $s_{x}$ and broadcast data $B(\Lambda,r)$:
\begin{eqnarray}	%(3.8)
U &=&X^{s_x\times L(\Lambda\cup\Theta\cup\{x\},x)}\nonumber\\[2pt]
&&\times\,\prod_{j\in\Lambda\cup\Theta}
M_j^{L(\Lambda\cup\Theta\cup\{x\},j)}\bmod p
\label{eq:cU}
\end{eqnarray}
where
\begin{equation}	%(3.9)
L(\Psi,w)=\sum_{t\in\Psi\setminus\{w\}}
\frac{t}{t-w}\bmod q\quad (\forall\,\Psi: set,\forall\,w:integer)
\label{eq:LIP}
\end{equation}
Since $M_j=g^{s_j\times r}\bmod p$ holds, the system secret key $S$ is
recovered in the exponent part of equation (\ref{eq:cU}), gathering
$k$ sets of secret keys.

\begin{table}[ht]	%Table~1.
\tbl{Comparison of five schemes.}
{\begin{tabular}{@{}c|c|c|c|c@{}}
\hline
&&&\multicolumn{2}{|c@{}}{}\\[-6pt]
&Flexibility &Reusability &\multicolumn{2}{c@{}}{Scalability}\\[3pt]
\cline{4-5}
&&&&\\[-6pt]
&&&Amount of &Key \\
&&&transmission &storage\\[3pt]
\hline
&&&&\\[-6pt]
MaSK &OK &OK &OK &OK\\[3pt]
HKDS &OK &OK &NG &NG\\[3pt]
FN &NG &OK &NG &NG\\[3pt]
MA &NG &NG &OK &OK\\[3pt]
KRS &NG &OK &OK &NG\\[3pt]
\hline
\end{tabular}\label{tab1}}
\end{table}

In summary, MaSK (Table~\ref{tab1}) is effective to implement quick
group key distribution with entity exclusion function when the group
scale\index{scale} $n$ is very large compared to the number of
excluded users $d$ and it works well for devices\index{devices} with
limited storage.\index{storage}

\section{How to Discover the Existence of a Clone}
\subsection{Security}
A type of attack likely to occur would be the ``one-shot attack,'' in
which an attacker reads out by some means the ID, the private key and
other information from the target\index{target} terminal and makes a
clone. After getting the above information, the attacker returns the
terminal without any change to the holder of the terminal. In other
words, the attacker does not use the original terminal at the very
moment when it tries to fool the center.

Figure~\ref{fig:shell} illustrates the shell model. Certificate
$\langle\langle EE\rangle\rangle CA_3$ is valid at time $t_1$ as all
three certificates are valid, but is invalid at time $t_2$ as
certificate $\langle\langle CA_2\rangle\rangle CA_1$ has expired.

\section*{Acknowledgments}
The content of this note is based on a lecture given at IMS of
National University of Singapore in September 2001. The author thanks
Professor Harald Niederreiter and IMS for providing such a nice
opportunity.

This work was partially supported by MEXT Grant-in-Aid for Scientific
Research 13224040.

%\begin{thebibliography}{0}    %means for 1-digit
\begin{thebibliography}{00}    %means for 2-digits
%\begin{thebibliography}{000}  %means for 3-digits
%1.
\bibitem{Anderson} R. Anderson, {\em Security Engineering}, John Wiley
\& Sons, pp.~352--353, 2001.

%2.
\bibitem{kn:AMM2} J. Anzai, N. Matsuzaki, and T. Matsumoto, ``A method
for masked sharing of group keys (3),'' {\em Technical Report of
IEICE}, ISEC99-38, pp.~1--8, 1999.

%3.
\bibitem{kn:AMM4} J. Anzai, N. Matsuzaki, and T. Matsumoto, ``A
flexible method for masked sharing of group keys,'' {\em IEICE
Trans. Fundamentals}, vol.~E84-A, no.~1, pp.~239--246, 2001.
Preliminary version appeared as J. Anzai, N. Matsuzaki, and
T. Matsumoto, ``A quick group key distribution scheme with entity
revocation,'' {\em Advances in Cryptology -- ASIACRYPT '99}, LNCS
vol.~1716, pp.~333--347, Springer-Verlag, 1999.

%4.
\bibitem{kn:AMM5} J. Anzai, N. Matsuzaki, and T. Matsumoto, ``Clone
discovery,'' to appear.

%5.
\bibitem{kn:B} S. Berkovits, ``How to broadcast a secret,'' {\em
Advances in Cryptology -- EURO-CRYPT '91}, LNCS vol.~547,
pp.~535--541, Springer-Verlag, 1992.

%6.
\bibitem{kn:BMS1} C. Blundo, L. Mattos, and D. Stinson, ``Generalized
Beimel-Chor schemes for broadcast encryption and interactive key
distribution,'' {\em Theoretical Computer Science}, 200(1-2),
pp.~313--334, 1998.

%7.
\bibitem{kn:BMS2} C. Blundo, L. Mattos, and D. Stinson, ``Trade-offs
between communication and storage in unconditionally secure schemes
for broadcast encryption and interactive key distribution,'' {\em
Advances in Cryptology --- CRYPTO '96}, LNCS vol.~1109, pp.~387--400,
Springer-Verlag, 1996.

%8.
\bibitem{kn:CGIM} R. Canetti, J. Garay, G. Itkis, D. Micciancio,
M. Naor, and B. Pinkas, ``Multicast security: a taxonomy and efficient
constructions,'' {\em Proceedings of \hbox{INFOCOM~'99}}, vol.~2,
pp.~708--716, 1999.

%9.
\bibitem{kn:CMN} R. Canetti, T. Malkin, and K. Nissim, ``Efficient
communication-storage tradeoffs for multicast encryption,''
{\em Advances in Cryptology --- EUROCRYPT '99}, LNCS vol.~1592,
pp.~459--474, Springer-Verlag, 1999.

%10.
\bibitem{Chor} B. Chor, A. Fiat, and M. Naor, ``Tracing traitors,''
{\em Advances in Cryptology --- CRYPTO '94}, LNCS vol.~839,
pp.~257--270, Springer-Verlag, 1994.

%11.
\bibitem{kn:CS} R. Cramer, V. Shoup, ``A practical public key
cryptosystem provably secure against adaptive chosen ciphertext
attack,'' {\em Advances in Cryptology --- CRYPTO '98}, LNCS vol.~1462,
pp.~13--25, Springer-Verlag, 1998.

%12.
\bibitem{kn:DF} Y. Desmedt, Y. Frankel, ``Threshold cryptosystems,''
{\em Advances in Cryptology --- CRYPTO '89}, LNCS vol.~435,
pp.~307--315, Springer-Verlag, 1989.

%13.
\bibitem{kn:FN} A. Fiat, M. Naor, ``Broadcast encryption,'' {\em
Advances in Cryptology --- CRYPTO '93}, LNCS vol.~773, pp.~480--491,
Springer-Verlag, 1993.

%14.
\bibitem{Kocher} P. Kocher, J. Jaffe, and B. Jun, ``Differential power
analysis,'' {\em Advances in Cryptology --- CRYPTO '99}, LNCS
vol.~1666, pp.~388--397, Springer-Verlag, 1999.

%15.
\bibitem{kn:KRS} R. Kumar, S. Rajagopalan, and A. Sahai, ``Coding
constructions for blacklisting problems without computational
assumptions,'' {\em Advances in Cryptology --- CRYPTO '99}, LNCS
vol.~1666, pp.~609--623, Springer-Verlag, 1999.

%16.
\bibitem{Kurosawa} K. Kurosawa, Y. Desmedt, ``Optimum traitor tracing
and asymmetric scheme,'' {\em Advances in Cryptology --- EUROCRYPT
'98}, LNCS vol.~1403, pp.~145--157, Springer-Verlag, 1998.

%17.
\bibitem{kn:LS} M. Luby, J. Staddon, ``Combinatorial bounds for
broadcast encryption,'' {\em Advances in Cryptology --- EUROCRYPT
'98}, LNCS vol.~1403, pp.~512--526, Springer-Verlag, 1998.

%18.
\bibitem{kn:MNTO} M. Mambo, A. Nishikawa, S. Tsujii, and E. Okamoto,
``Efficient secure broadcast communication systems,''
{\em Technical Report of IEICE}, ISEC93-34, pp.~21--31, 1993.

%19.
\bibitem{Matsushita} T. Matsushita, Y. Watanabe, K. Kobara, and
H. Imai, ``A sufficient \hbox{content} distribution scheme for mobile
subscribers,'' {\em Proceedings of International Symposium on
Information Theory and Its Applications, ISITA 2000}, pp.~497--500,
2000.

%20.
\bibitem{kn:MA} N. Matsuzaki, J. Anzai, ``Secure group key
distribution schemes with terminal revocation,'' {\em Proceedings of
JWIS'98} ({\em IEICE Technical Report}, ISEC98-52), pp.~37--44, 1998.

%21.
\bibitem{kn:MOV} A. Menezes, P. van Oorschot, and S. Vanstone, {\em
Handbook of Applied Cryptography}, CRC Press, pp.~113--114, 1997.

%22.
\bibitem{kn:NR} K. Nyberg, R.A. Rueppel, ``Message recovery for
signature schemes based on the discrete logarithm problem,'' {\em
Advances in Cryptology --- EUROCRYPT '94}, LNCS vol.~950,
pp.~182--193, Springer-Verlag, 1995.

%23.
\bibitem{RAO} J.R. Rao, P. Rohatgi, H. Scherzer, and S. Tinguely,
``Partitioning attacks: or how to rapidly clone some GSM cards,'' {\em
IEEE Symposium on Security and Privacy}, 2002.
\eject

%24.
\bibitem{kn:S} A. Shamir, ``How to share a secret,'' {\em
Communications of ACM}, vol.~22, no.~11, pp.~612--613, 1979.

%25.
\bibitem{kn:DSA} U. S. Dept. of Commerce/National Institute of
Standards and Technology, ``Digital signature standard,'' {\em Federal
Information Processing Standards Publication 186-1}, 1998.

%26.
\bibitem{kn:WHA} D. Wallner, E. Harder, and R. Agee, ``Key management
for multicast: issues and architectures,'' RFC2627, IETF, June 1999.

%27.
\bibitem{kn:WGL} C. Wong, M. Gouda, and S. Lam, ``Secure group
communications using Key graphs,'' {\em Proceedings of ACM SIGCOMM
'98}, 1998. Also available as Technical Report TR 97-23, Department of
Computer Science, The University of Texas at Austin, July 1997.
\end{thebibliography}

\end{document}
